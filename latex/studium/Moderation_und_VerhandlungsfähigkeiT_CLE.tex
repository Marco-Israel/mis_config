%-------------------------------------------------------------------------------
% @author    Marco Israel (MI)
% @date      2020-12
% @authors   Last changed by:  Marco Israel - 2020.
%
%
% @brief     German DinBrief template
% @details   German DinBrief template
%
%
%  @example
%  "mkpdf -f letter_template.tex"
%
%  @copyright     Available under the MIT License.
%
%			Copyright (C) 2020 Marco Israel (MI).			All rights reserved.
%
% Permission is hereby granted, free of charge, to any person obtaining a copy
% of this software and associated documentation files (the "Software"), to deal
% in
% the Software without restriction, including without limitation the rights to
% use, copy, modify, merge, publish, distribute, sublicense, and/or sell copies
% of the Software, and to permit persons to whom the Software is furnished to
% do so, subject to the following conditions:
%
% The above copyright notice and this permission notice shall and MUST BE
% included in all copies or substantial portions of the Software.
%
% THE SOFTWARE IS PROVIDED "AS IS", WITHOUT WARRANTY OF ANY KIND, EXPRESS OR
% IMPLIED, INCLUDING BUT NOT LIMITED TO THE WARRANTIES OF MERCHANTABILITY,
% FITNESS FOR A PARTICULAR PURPOSE AND NONINFRINGEMENT. IN NO EVENT SHALL THE
% AUTHORS OR COPYRIGHT HOLDERS BE LIABLE FOR ANY CLAIM, DAMAGES OR OTHER
% LIABILITY, WHETHER IN AN ACTION OF CONTRACT, TORT OR OTHERWISE, ARISING FROM,
% OUT OF OR IN CONNECTION WITH THE SOFTWARE OR THE USE OR OTHER DEALINGS IN
% THE SOFTWARE.
%-------------------------------------------------------------------------------

%-------------------------------------------------------------------------------
%	DOCUMENT CONFIGURATIONS
%-------------------------------------------------------------------------------

\documentclass[
    version=last,           % use last release of scrlttr2 package
    DIV=13,                 % ?
    BCOR=0mm,               % ?
    paper=a4,               % paper size
    fontsize=12pt,          % fontsize
    firsthead=on,           % display header on first page
    firstfoot=on,           % display footer on first page
    pagenumber=on,i         % position of the page number
    parskip=half,           % Use indent instead of skip, half, false
    enlargefirstpage=,      % more space on first page
    firsthead=on,           % Display the adress
    fromrule=afteraddress,  % separate the address with a line in letter he
    priority=off,           % print a priority onto the document
    backaddress=true,       % Adress to send back
    refline=dateright,      % location of the date
	fromalign=right,	    % Aligns the from address to the right
    fromemail=on,i          % turn on email of sender
    fromurl=on,             % print URL of sender
    frombank=on,
    fromphone=on,           % turn on phone of sender
    frommobilephone=on      % Mobile Phone number
    fromlogo=on,            % turn on logo of sender
    addrfield=on,           % address field for envelope with window, on or true
    subject=untitled,  % placement of subject, beforeopening or titled
    foldmarks=off,          % print foldmarks
    numericaldate=off,      % display date in numbers only
	pagenumber=right,	        % Set page numbers from page 2 onwards
	parskip=half,	        % Separates paragraphs with some whitespace,
    headsep=false,          % Seperatorline in the header
    footsepline=true,       % Seperatorline in the footer
			%use parskip=full for more space or comment out to return to default
    foldmarks=off,		    % Prints small fold marks on the left of the page
	]{scrlttr2}

% Set Font: sans serif Latin Modern
\usepackage{lmodern}
\usepackage[T1]{fontenc} % For extra glyphs (accents, etc)
\usepackage{stix} % Use the Stix font by default
\usepackage{lmodern}
\usepackage[utf8]{inputenc}
%\usepackage[english]{babel}
\usepackage[ngerman,english]{babel}
\usepackage{url}
\usepackage{csquotes}
\usepackage{listings}
\usepackage{verbatim}
\usepackage{graphicx}
\usepackage{tabularx}
\renewcommand*{\raggedsignature}{\raggedright} %Signatur rechtsbündig

% Set Page layout:
\usepackage{changepage}
    %\changepage{text height}{text width}{even-side margin}
    %{odd-side margin}{column sep.}
    %{topmargin}{headheight}{headsep}{footskip}
\changepage{}{}{}{}{}{}{}{}{}

%-------------------------------------------------------------------------------
%	YOUR INFORMATION AND LETTER DATE
%-------------------------------------------------------------------------------

\KOMAoptions{foldmarks=true,
foldmarks=false,
fromurl=false,
fromemail=true,
fromphone=false,
fromfax=false,
fromalign=right,
fromrule=off,
footsepline=on,
fromlogo=true,
headsepline = true,
footsepline = true
%fromrule=afteraddress%  % Trennlinie unter dem Briefkopf
}

%\setkomavar{fromlogo}{\parbox[b]{8cm}{\usekomafont{fromaddress}%
%        {\mbox{\LARGE \bfseries Embedded System Design GmbH}}
%        \smallskip}
%}
%

\setkomavar*{urlseparator}{}
\setkomavar{urlseparator}{}
\setkomavar{frombank}{Eine Bank\\BLZ~123\,45\,678\\Kto~123456789}
\setkomavar{fromname}{Marco Israel} % Your name used in the from address
\setkomavar{fromurl}{www.marcois.eu}
\setkomavar{fromaddress}{Am Wickenkamp 38\\ 32351 Stemwede}
\setkomavar{fromemail}{Marco-Israel@online.de}
\setkomavar{signature}{Marco Israel}
\setkomavar{fromphone}{05773 00000000}
\setkomavar{frommobilephone}{0175 00000}
%\setkomavar{place}{Stemwede}
%\includegraphics[⟨options⟩]{⟨file⟩}
\firstfoot{}
\nextfoot{}


%-------------------------------------------------------------------------------
%  HEADER SECTION
%-------------------------------------------------------------------------------
%\firsthead{\centering
%        {\mbox{\LARGE \bfseries Embedded System Design GmbH}}
%}
%

%-------------------------------------------------------------------------------
%  FOOTER SECTION
%-------------------------------------------------------------------------------
%\firstfoot{%
%\centering
%{\renewcommand{\\}{\ {\large\textperiodcentered}\ }
%\small\usekomavar{frombank}
%}%
%}
%




%-------------------------------------------------------------------------------

\begin{document}

%-------------------------------------------------------------------------------
%	ADDRESSEE
%-------------------------------------------------------------------------------

% Addressee name and address
\begin{letter} {Wilhelm Büchner Hochschule \\
Hilpertstr. 31\\
64295 Darmstad}

\setkomavar*{yourref}{AufgabenCode}
\setkomavar{yourref}{CLE03-XX1-N01}		% Ihr Zeichen
\setkomavar*{yourmail}{HeftKürzel} 	% Ihr Schreiben vom
\setkomavar{yourmail}{CLE03XX 1} 	% Ihr Schreiben vom
\setkomavar*{myref}{Auflage}       	% Unser Zeichen
\setkomavar{myref}{0111 N01}       	% Unser Zeichen
\setkomavar*{customer}{Matrikel-Nr}
\setkomavar{customer}{580201} 	% Kundennummer
\setkomavar*{invoice}{StudiengangsNr.}   	% Rechnungsnummer
\setkomavar{invoice}{1640}   	% Rechnungsnummer
\setkomavar{place}{Stemwede}		% Ort
\setkomavar{date}{\today}			% Datum

\setkomavar{subject}{Einsendeaufgaben Typ A }

\opening{Sehr geehrte(r) Herr / Frau}

Guten Tag,

im Anhang die Lösungen für o.g. Einsendeaufgabe Typ A,
\\
\begin{itemize}
    \item Multiple Choice
        \begin{enumerate}
            \item \textbf{Ziele der Moderation} : D \\
            \item \textbf{Facilitator und Facilitation} : D\\
            \item \textbf{Verhandlungen} : D\\
            \item \textbf{Harvard Negotiation Concept} : D \\
        \end{enumerate}
            \vspace{1cm}
    \item Textaufgaben
        \begin{enumerate}
        \setcounter{enumi}{4}
        \item \textbf{Open Space Technology} von Harrison Owen \\
                Der Ansatz der Open Space Technology von Harrison Owen ist eine
                Moderatsions Methode zur Unterstützung, Gliederung und
                Strukturierung von Großkonferenzen. Dabei ist die Offenheit
                hinsichtlich Ablaufplanung und Themen wichtig. Hauptziel ist das
                schnelle und effektive Bearbeiten meist komplexer Themen in
                kurzer Zeit ohne eine konkrete Vorplanung oder kontinuierliche
                Anleitung durch Außenstehende. Dabei obliegen Inhalte, Ideen oder
                Projekte den Interessen, Wünschen und Vorstellungen der
                Teilnehmer wozu sie aufgefordert sind, diese (in Bezug zum
                General-/Haupt-/Oberthema) einzubringen. So wird die
                Leidenschaft zur Umsetzung als Hauptantriebskraft zur
                produktiven und Verantwortungs Mitarbeit hervorgerufen (Owen,
                2011).
                \\
                Die \textbf{vier wichtigsten Grundsätze und Gesetzte} sind nach
                Owen:
                \begin{itemize}
                    \item Wer auch immer kommt, es sind die richtigen Leute: \\
                        Jeder Teilnehmer hat etwas positives beizutragen, auch
                        wenn es nicht oder insbesondere wenn es nicht dem
                        standard entspricht.
                    \item Was auch immer geschieht, ist das einzige, was
                        geschehen kann. \\
                        Dies beschreibt, dass das 'Jetzt' und die aktuelle
                        Situation bei der Gedankenfindung betrachtet werden soll
                        und nicht von Eventualitäten und Prognosen ausgegangen
                        werden soll. Die aktuelle Situation und die aktuellen
                        Mittel zählen, oder solche die definitiv realistisch
                        beschafft werden können.
                    \item Es fängt an, wenn die Zeit reif ist: \\
                        Zeit sollte nur in gewissen Maße eine Rolle spielen und
                        ansonsten keinen weiteren Einfluss haben, der an ansonsten
                        durch Zeitdruck Einfluss auf die Ergebnisse haben kann.
                    \item Vorbei ist vorbei: Diese Regel beschreibt zum einen,
                        sich von alten Denkmustern zu trennen und sich
                        alternativen Formen der Zusammenarbeiten zu öffnen. Zum
                        anderen beschreibt es, das Zusammenarbeiten beendet
                        werden sollten, wenn sie festgefahren und nicht mehr
                        effektiv sind.
                \end{itemize}
                Sowie das \textbf{Gesetz der zwei Füße}: Jeder Teilnehmer einer
                Gruppe soll entweder einen eigenen Beitrag leisten, aus den
                Beiträgen anderer etwas (bereitwillig) lernen (wollen) oder die
                Gruppe verlassen (und einer neuen Gruppe beitreten).

            \vspace{1cm}
            \item \textbf{Moderatsions-Techniken} vs \textbf{Moderatsions-Werkzeuge}:
                Der Unterschied liegt darin begründet, dass:
                Moderations-Techniken sind die \textit{Art und Weise} oder auch
                das die \textit{Vorgehensweisen} und / der einzelnen Moderations-Schritte:
                Etwa wie ein Moderator durch einen Gruppenprozess steuert,
                Diskussionen, Ideen und Ergebnisse hervorruft. Moderations-Werkzeuge
                hingegen sind reale Werkzeuge und Mittel beispielweise um Ideen,
                Inhalte oder Prozessschritte zu Visualisieren oder zu Strukturieren.
            \begin{description}
        \item[Werkzeuge:] \hfill \\
            \begin{itemize}
                \item Pinnwände
                \item Flipcharts
                \item Beamer
                \item Smart-Boards
            \end{itemize}
        \item[Methoden:] \hfill \\
            \begin{itemize}
                \item Moderations-Plan
                \item Brainstorming
                \item Clustering
                \item Erwartungsabfragen
            \end{itemize}
            \end{description}

            \vspace{1cm}
            \item  \textbf{Vier Verhandlungssituationen}
                \begin{description}
                \item[Competition (I win, you lose) ]
                 Das eigene Interesse
                    steht nahezu ausschließlich im Vordergrund. Dies erfordert
                    einen harten Verhandlungsstiel und viel Selbstbewusstsein.
                \item[Collaboration (I win, you win) ]
                    Das erreichen der Ziele ist ebenso wichtig wie das Aufrecht
                    erhalten der Beziehungen zum Gegenüber und es soll ein
                    Resultat erzielt werden, mit dem beide Seiten langfristig
                    zufrieden sind, weil die Beziehung einen höheren Mehrwert
                    hat, als das Maximum in der zu verhandelnde Sache.
                    Die Stategie lautet, das beiden Seiten die Bedeutung der
                    Beziehung im Vergleich zum Differenzwert der Sache bewusst
                    ist oder wird.
                \item[Accommodation (I lose, you win) ]
                    Die Beziehung zum (Verhandlungs) Partner ist mehr wehrt als
                    der Wert der zu verhandelnden Sache. Aus diesem Grund werden
                    die eigenen Absichten und der eigene Mehrwert der zu
                    verhandelten Sache der Beziehung zuliebe minimiert.
                    Die Strategie auf einer Seite ist der Beziehung zuliebe sehr
                    weich und es wird der Wert der Sache nahezu vollkommen
                    vernachlässige.
                \item[Avoidance (I lose, you lose) ]
                    Eine Verhandlung kommt praktisch nicht zustande. Weder die
                    Beziehung zum Partner hat einen Wert für eine Seite, noch
                    die zu verhandelnde Sache hat einen Wert (für eine der beiden
                    Seiten). Ohne eine Verhandlung, gibt es auch keine
                    Verhandlungsstrategie, außer zunächst den Wert der Sache
                    und/oder des Partners zu erkennen.
                \end{description}

            \vspace{1cm}
            \item Grundaspekte einer Verhandlung nach dem \textbf{ Harvard-Konzept}
                Das klassische Harvard Modell begründet sich in der
                \textit{harten} und \textit{weichen} Verhandlung: Hat in der
                Sache, aber fair (weich) zum Menschen. Die Erweiterung, das
                Harvard Negotiation Concept basiert auf den o.g. \textit{vier
                Verhandlungssituationen}. Es basiert auf vier Grundaspekte einer
                Verhandlung:
                \begin{description}
                \item [Menschen und Probleme getrennt voneinander behandeln.]
                    Bei Verhandlungen sotten beide Seiten Wertevorstellungen
                    und Emotionen trennen. Herzu muss man in
                    der Regel die Gefühle und Emotionen der Gegenseite zu
                    mindestens in Teilen verstehen (sich hineinversetzten).
                    Diese sollten also im Vorfeld kommuniziert und von der
                    Gegenseite akzeptiert und nachvollzogen werden.
                    Zwar können Ehemutationen in eine Verhandlung auch
                    vorteilhaft sein (mindestens für eine Seite). Jedoch können
                    Emotionen ebenso schnell überkochen und in einen Streit ohne
                    Ergebnis eskalieren. Die Gegenseite und seine Emotionen zu
                    Verstehen hilft auch, die wahren Interessen einer Sache einer
                    Seite zu begründen mögliche Alternativen ausarbeiten zu
                    können (s.u jeweils).

                \item [Auf Interesse konzentrieren, nicht auf Position.]
                    Das Interesse an einer Sache ist der Motor bzw. die
                    Motivation einer Position gegenüber einer Sache.  Daher ist
                    es wichtig, sowohl das eigene Interesse, als auch das
                    Interesse der Gegenseite zu hinterfragen um
                    beide Postionen verstehen zu können. Hierzu sind
                    Wertneutrale Fragetechniken sinnvoll, etwa nach dem "Warum"
                    oder dem "Warum nicht".
                \item [Entwickeln von Verhandlungsoptionen zum beidseitigen Vorteil.]
                    Durch \\ Kreativ- Techniken sollen alternative
                    Verhandlungsoptionen (Ergebnisse) ermittelt werden, die für
                    beide Seiten ok sind. Beispielsweise indem sie alternativem
                    zur eigentlichen (vollen) Sache bieten (welche vorher ggf.
                    noch nicht erkannt wurden) aber ebenso einen Mehrwert bieten
                    oder den Verlust ausgleichen. Hierzu stehen verschiedene
                    Techniken zu Verfügung, beispielsweise die
                    Kreisdiagramm-Technik zur Entwicklung von Entscheidungsoptionen.

                \item [Auf Anwendung neutraler Beurteilungskriterien bestehen.]
                    Die Kriterien zur Beurteilung einer Sache sollen möglichst
                    objektiv sein (etwa auf Basis von Zahlen, Daten,
                    Vorschriften), und vor allem keine (persönliche) Werte
                    enthalten. Ein Streitfall muss dahingehend umgewandelt werden,
                    dass er auf objektiver Ebene ausgetragen wird.
                \end{description}
        \end{enumerate}
\end{itemize}
\vspace{1cm}
Mit freundlichen Grüßen \\
\\
    \includegraphics[scale=0.70]{../unterschrift.png}
    \\
    Marco Israel



%\encl{Myattachments} % Attached documents

%-------------------------------------------------------------------------------
%   Postskriptum
%
%\ps\ PS:\ someText
%-------------------------------------------------------------------------------


%-------------------------------------------------------------------------------
%  Anlagen
%\setkomavar*{enclseparator}{Anlage}
%\encl{%
%  Anlage 1\\
%  Anlage 2%
%}
%-------------------------------------------------------------------------------


%-------------------------------------------------------------------------------
%   Verteiler
%\setkomavar*{ccseparator}{Kopie an}
%\cc{%
%  Verteiler 1\\
%  Verteiler 2%
%}
%-------------------------------------------------------------------------------

\end{letter}

\end{document}

%--- EOF -----------------------------------------------------------------------
%
%
