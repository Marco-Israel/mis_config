%-------------------------------------------------------------------------------
% @author    Marco Israel (MI)
% @date      2020-12
% @authors   Last changed by:  Marco Israel - 2020.
%
%
% @brief     German DinBrief template
% @details   German DinBrief template
%
%
%  @example
%  "mkpdf -f letter_template.tex"
%
%  @copyright     Available under the MIT License.
%
%			Copyright (C) 2020 Marco Israel (MI).			All rights reserved.
%
% Permission is hereby granted, free of charge, to any person obtaining a copy
% of this software and associated documentation files (the "Software"), to deal
% in
% the Software without restriction, including without limitation the rights to
% use, copy, modify, merge, publish, distribute, sublicense, and/or sell copies
% of the Software, and to permit persons to whom the Software is furnished to
% do so, subject to the following conditions:
%
% The above copyright notice and this permission notice shall and MUST BE
% included in all copies or substantial portions of the Software.
%
% THE SOFTWARE IS PROVIDED "AS IS", WITHOUT WARRANTY OF ANY KIND, EXPRESS OR
% IMPLIED, INCLUDING BUT NOT LIMITED TO THE WARRANTIES OF MERCHANTABILITY,
% FITNESS FOR A PARTICULAR PURPOSE AND NONINFRINGEMENT. IN NO EVENT SHALL THE
% AUTHORS OR COPYRIGHT HOLDERS BE LIABLE FOR ANY CLAIM, DAMAGES OR OTHER
% LIABILITY, WHETHER IN AN ACTION OF CONTRACT, TORT OR OTHERWISE, ARISING FROM,
% OUT OF OR IN CONNECTION WITH THE SOFTWARE OR THE USE OR OTHER DEALINGS IN
% THE SOFTWARE.
%-------------------------------------------------------------------------------

%-------------------------------------------------------------------------------
%	DOCUMENT CONFIGURATIONS
%-------------------------------------------------------------------------------

\documentclass[
    version=last,           % use last release of scrlttr2 package
    DIV=13,                 % ?
    BCOR=0mm,               % ?
    paper=a4,               % paper size
    fontsize=12pt,          % fontsize
    firsthead=on,           % display header on first page
    firstfoot=on,           % display footer on first page
    pagenumber=on,i         % position of the page number
    parskip=half,           % Use indent instead of skip, half, false
    enlargefirstpage=,      % more space on first page
    firsthead=on,           % Display the adress
    fromrule=afteraddress,  % separate the address with a line in letter he
    priority=off,           % print a priority onto the document
    backaddress=true,       % Adress to send back
    refline=dateright,      % location of the date
	fromalign=right,	    % Aligns the from address to the right
    fromemail=on,i          % turn on email of sender
    fromurl=on,             % print URL of sender
    frombank=on,
    fromphone=on,           % turn on phone of sender
    frommobilephone=on      % Mobile Phone number
    fromlogo=on,            % turn on logo of sender
    addrfield=on,           % address field for envelope with window, on or true
    subject=untitled,  % placement of subject, beforeopening or titled
    foldmarks=off,          % print foldmarks
    numericaldate=off,      % display date in numbers only
	pagenumber=right,	        % Set page numbers from page 2 onwards
	parskip=half,	        % Separates paragraphs with some whitespace,
    headsep=false,          % Seperatorline in the header
    footsepline=true,       % Seperatorline in the footer
			%use parskip=full for more space or comment out to return to default
    foldmarks=off,		    % Prints small fold marks on the left of the page
	]{scrlttr2}

% Set Font: sans serif Latin Modern
\usepackage{lmodern}
\usepackage[T1]{fontenc} % For extra glyphs (accents, etc)
\usepackage{stix} % Use the Stix font by default
\usepackage{lmodern}
\usepackage[utf8]{inputenc}
%\usepackage[english]{babel}
\usepackage[ngerman,english]{babel}
\usepackage{url}
\usepackage{csquotes}
\usepackage{listings}
\usepackage{verbatim}
\usepackage{graphicx}
\usepackage{tabularx}
\renewcommand*{\raggedsignature}{\raggedright} %Signatur rechtsbündig

\usepackage{array}
\usepackage{makecell}


% Set Page layout:
\usepackage{changepage}
    %\changepage{text height}{text width}{even-side margin}
    %{odd-side margin}{column sep.}
    %{topmargin}{headheight}{headsep}{footskip}
\changepage{}{}{}{}{}{}{}{}{}

%-------------------------------------------------------------------------------
%	YOUR INFORMATION AND LETTER DATE
%-------------------------------------------------------------------------------

\KOMAoptions{foldmarks=true,
foldmarks=false,
fromurl=false,
fromemail=true,
fromphone=false,
fromfax=false,
fromalign=right,
fromrule=off,
footsepline=on,
fromlogo=true,
headsepline = true,
footsepline = true
%fromrule=afteraddress%  % Trennlinie unter dem Briefkopf
}

%\setkomavar{fromlogo}{\parbox[b]{8cm}{\usekomafont{fromaddress}%
%        {\mbox{\LARGE \bfseries Embedded System Design GmbH}}
%        \smallskip}
%}
%

\setkomavar*{urlseparator}{}
\setkomavar{urlseparator}{}
\setkomavar{frombank}{Eine Bank\\BLZ~123\,45\,678\\Kto~123456789}
\setkomavar{fromname}{Marco Israel} % Your name used in the from address
\setkomavar{fromurl}{www.marcois.eu}
\setkomavar{fromaddress}{Am Wickenkamp 38\\ 32351 Stemwede}
\setkomavar{fromemail}{Marco-Israel@online.de}
\setkomavar{signature}{Marco Israel}
\setkomavar{fromphone}{05773 00000000}
\setkomavar{frommobilephone}{0175 00000}
%\setkomavar{place}{Stemwede}
%\includegraphics[⟨options⟩]{⟨file⟩}
\firstfoot{}
\nextfoot{}


%-------------------------------------------------------------------------------
%  HEADER SECTION
%-------------------------------------------------------------------------------
%\firsthead{\centering
%        {\mbox{\LARGE \bfseries Embedded System Design GmbH}}
%}
%

%-------------------------------------------------------------------------------
%  FOOTER SECTION
%-------------------------------------------------------------------------------
%\firstfoot{%
%\centering
%{\renewcommand{\\}{\ {\large\textperiodcentered}\ }
%\small\usekomavar{frombank}
%}%
%}
%




%-------------------------------------------------------------------------------

\begin{document}

%-------------------------------------------------------------------------------
%	ADDRESSEE
%-------------------------------------------------------------------------------

% Addressee name and address
\begin{letter} {Wilhelm Büchner Hochschule \\
Hilpertstr. 31\\
64295 Darmstad}

\setkomavar*{yourref}{AufgabenCode}
\setkomavar{yourref}{CLE02-XX1-K02}		% Ihr Zeichen
\setkomavar*{yourmail}{HeftKürzel} 	% Ihr Schreiben vom
\setkomavar{yourmail}{CLE02XX} 	% Ihr Schreiben vom
\setkomavar*{myref}{Auflage}       	% Unser Zeichen
\setkomavar{myref}{0119K02}       	% Unser Zeichen
\setkomavar*{customer}{Matrikel-Nr}
\setkomavar{customer}{580201} 	% Kundennummer
\setkomavar*{invoice}{StudiengangsNr.}   	% Rechnungsnummer
\setkomavar{invoice}{1640}   	% Rechnungsnummer
\setkomavar{place}{Stemwede}		% Ort
\setkomavar{date}{\today}			% Datum

\setkomavar{subject}{Einsendeaufgaben Typ A }

\opening{Sehr geehrte(r) Herr / Frau}

Guten Tag,

im Anhang die Lösungen für o.g. Einsendeaufgabe Typ A,
\\
\begin{itemize}
    \item Multiple Choice
        \begin{enumerate}
        \item \textbf{CSCW:} D
        \item \textbf{Web 2.0:} A. (Wobei in Zeiten des Dynamischen  Webs im
            Gegensatz zum statischen Web ist wohl 4. nicht mehr richtig.
            Aber das Studienheft ist sicher auch noch aus Zeiten des statischen
            webs und alle anderen Punkte sind irgendwie auch richtig.).
       \item \textbf{Social-Sharing-Plattformen}: A
       \item \textbf{Anforderungen an Enterprise 2.0 nach MacAfee:} B


        \end{enumerate}
            \vspace{1cm}
    \item Textaufgaben
        \begin{enumerate}
        \setcounter{enumi}{2}
    \item \textbf{Ansatzpunkte der Vorwärts- und Rückwerts CSCW}
        Bei der Vorwärtsinterpretation steht der Computer (C) als
        Unterstützungsmedium (S) am Anfang der Betrachtung um eine Form
        der Zusammenarbeit / Kooperation (C) bei Arbeitsaufgaben (W) zu
        unterstützen. Der Fokus liegt auf der Technologie mit der Aufgaben
        erfüllt werden können.\\

        Bei der Rückwärtsinterpretation ist die Arbeitsaufgabe (W) Ausgangspunkt,
        welche kooperativ und arbeitsteilig durchgeführt wird (C) und
        Unterstützung durch Technologie erfährt (C). Der Fokus liegt auf der
        Aufgabe, nicht auf der eingesetzten Technologie. \\

        Das 3-K Modell beschreibt dabei die Art der unterstützten Interaktion
        zwischen den Teammitgliedern durch die Technik (C) und kann auf Basis
        von Kommunikation, Koordination oder Kooperation erfolgen. Die
        eingesetzten, unterstützenden Technologien lassen sich in Systemklassen
        einteilen; wie \textit{Kommunikation, gemeinsame Informationsräume,
        Workflow Management} oder \textit{Workgroup Computing}.
            \vspace{1cm}
        \item \textbf{Enterprise 2.0 und die Funktionen der sechs Komponenten}
            Enterprise 2.0 beschreibt nach MacAfee: den unternehmerischen
            Einsatz von Social Software. Sie sammeln nicht nur Wissen, sondern
            auch Praktiken und Ergebnisse der Mitarbeiter dokumentieren. Dabei
            sollen Enterprise 2.0 sechs nachfolgende Kriterien erfüllen, die
            auch als \textit{SALSE-Kriterien} bezeichnet werden.

            \begin{description}
            \item[Search:] Schlagwortsuche ermöglichen das finden relevanter,
                gesuchter Informationen und Inhalte.
            \item[Links:] Es sollen selbst Links auf relevante Inhalte erstellt
                werden, bzw. Inhalte untereinander verlinkt werden können um so
                die Navigation und Suche zu erleichtern.
                \item[Authoring] Mitarbeiter sollen die Möglichkeit haben, Beiträge
                    zu verfassen und und eigne wie andere zu editieren. So ist
                    das System (Intranet, Internet, ...) größtmöglich aktuell
                    und wachsend,
                \item[Tags:] Das Tagen von Inhalten bedeutet in Kategorisieren
                    von Inhalten. Beiträge und Vorschläge können so anhand von
                    Tags gefunden und gefiltert werden.
                \item[Extensions]
                    Extension sind Erweiterungen von Tags. Sie liefern
                    Suchergebnisse und Vorschläge aufgrund des (Surf-) Verhaltens
                    des Anwenders und seiner Interessen
                \item[Signals:] Signale wie Feeds informieren Interessenten /
                    Abonnenten der Feeds über Änderungen in Inhalten bestimmter
                    Seiten oder ganzer Kategorialen (Tags-Gruppen).
            \end{description}

            \vspace{1cm}

        \item \textbf{Formen von Awareness, Informationstransparenz und wie sich
            diese auf Social Software auswirkt}
            Die Erkenntnis der Mitarbeiter, dass sie Teil einer Gemeinschaft
            sind, ist ein wesentliches Kriterium für eine erfolgreiche
            Zusammenarbeit. Das wissen um den Andren, seiner Interessen und
            Stärken sowie über gemeinsame Artefakte bildet eine Basis für
            Zusammenarbeit und weiteren Austausch.


            \vspace{1cm}
        \item \textbf{Sieben Paradigmen zur Charakterisierung des Web 2.0}
            \begin{description}
                \item[The Web is a Platform]:
                    Das Web 2.0 ist eine offene Plattform, welche keinem alleine
                    gehört, sondern sich gegenseitig durch unterschiedliche
                    Dienste, freie Standards und Inhalte ergänzt und erweitert.
                    Solche werden geteilt und wiederverwendet.
                \item[Harnessing Collective Intelligence]:
                \item[Data is the Next \textit{Intel Inside} ]:
                    Den Kern von Webapplikationen bilden die Daten (Inhalte und
                    Informationen) selbst sowie drumherum die Mechanismen diese
                    Abzulegen, Aufzubereiten, darzustellen, zu speichern, zu
                    kategorisieren und aufzufinden.
                \item[Lightweigt Programming Models]:
                \item[End of Software Release Cycle]:
                    Webanwendungen werden zunehmend eine Dienstleistung, die
                    sich stetig dem Nutzerverhalten und seinen (An-)Forderungen
                    und Wünschen anpasst. Es existiert somit kein endgültiger
                    Software release mehr.
                \item[Software above the Level of a Single Device]:
                    Der Nutzer soll unabhängig einer bestimmten Hardware oder
                    Software werden. Er soll von überall und zu jederzeit
                    zugriff auf seine Daten und auf Informationen haben
                    unabhängig der ihm aktuell zur Verfügung stehenden
                    technischen Möglichkeiten. Beispielsweise: Datenspeicherung
                    in einer globalen Cloud und Zugriff auf diese wie
                    unterschiedlicher Geräte wie PC, Tablet,Smartphone.
                \item[Rich User Experiences]:
                    Durch verschiedene Technologien (z.B. Ajax) werden
                    Webapplikationen mit interaktiven Benutzeroberflächen
                    möglich, die klassischen Desktopanwendungen ebenbürtig sind.
            \end{description}


        \end{enumerate}
\end{itemize}
\vspace{1cm}
Mit freundlichen Grüßen \\
\\
    \includegraphics[scale=0.70]{../unterschrift.png}
    \\
    Marco Israel



%\encl{Myattachments} % Attached documents

%-------------------------------------------------------------------------------
%   Postskriptum
%
%\ps\ PS:\ someText
%-------------------------------------------------------------------------------


%-------------------------------------------------------------------------------
%  Anlagen
%\setkomavar*{enclseparator}{Anlage}
%\encl{%
%  Anlage 1\\
%  Anlage 2%
%}
%-------------------------------------------------------------------------------


%-------------------------------------------------------------------------------
%   Verteiler
%\setkomavar*{ccseparator}{Kopie an}
%\cc{%
%  Verteiler 1\\
%  Verteiler 2%
%}
%-------------------------------------------------------------------------------

\end{letter}

\end{document}

%--- EOF -----------------------------------------------------------------------
%
%
