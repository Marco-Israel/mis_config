%-------------------------------------------------------------------------------
% @author    Marco Israel (MI)
% @date      2020-12
% @authors   Last changed by:  Marco Israel - 2020.
%
%
% @brief     German DinBrief template
% @details   German DinBrief template
%
%
%  @example
%  "mkpdf -f letter_template.tex"
%
%  @copyright     Available under the MIT License.
%
%			Copyright (C) 2020 Marco Israel (MI).			All rights reserved.
%
% Permission is hereby granted, free of charge, to any person obtaining a copy
% of this software and associated documentation files (the "Software"), to deal
% in
% the Software without restriction, including without limitation the rights to
% use, copy, modify, merge, publish, distribute, sublicense, and/or sell copies
% of the Software, and to permit persons to whom the Software is furnished to
% do so, subject to the following conditions:
%
% The above copyright notice and this permission notice shall and MUST BE
% included in all copies or substantial portions of the Software.
%
% THE SOFTWARE IS PROVIDED "AS IS", WITHOUT WARRANTY OF ANY KIND, EXPRESS OR
% IMPLIED, INCLUDING BUT NOT LIMITED TO THE WARRANTIES OF MERCHANTABILITY,
% FITNESS FOR A PARTICULAR PURPOSE AND NONINFRINGEMENT. IN NO EVENT SHALL THE
% AUTHORS OR COPYRIGHT HOLDERS BE LIABLE FOR ANY CLAIM, DAMAGES OR OTHER
% LIABILITY, WHETHER IN AN ACTION OF CONTRACT, TORT OR OTHERWISE, ARISING FROM,
% OUT OF OR IN CONNECTION WITH THE SOFTWARE OR THE USE OR OTHER DEALINGS IN
% THE SOFTWARE.
%-------------------------------------------------------------------------------

%-------------------------------------------------------------------------------
%	DOCUMENT CONFIGURATIONS
%-------------------------------------------------------------------------------

\documentclass[
    version=last,           % use last release of scrlttr2 package
    DIV=13,                 % ?
    BCOR=0mm,               % ?
    paper=a4,               % paper size
    fontsize=12pt,          % fontsize
    firsthead=on,           % display header on first page
    firstfoot=on,           % display footer on first page
    pagenumber=on,i         % position of the page number
    parskip=half,           % Use indent instead of skip, half, false
    enlargefirstpage=,      % more space on first page
    firsthead=on,           % Display the adress
    fromrule=afteraddress,  % separate the address with a line in letter he
    priority=off,           % print a priority onto the document
    backaddress=true,       % Adress to send back
    refline=dateright,      % location of the date
	fromalign=right,	    % Aligns the from address to the right
    fromemail=on,i          % turn on email of sender
    fromurl=on,             % print URL of sender
    frombank=on,
    fromphone=on,           % turn on phone of sender
    frommobilephone=on      % Mobile Phone number
    fromlogo=on,            % turn on logo of sender
    addrfield=on,           % address field for envelope with window, on or true
    subject=untitled,  % placement of subject, beforeopening or titled
    foldmarks=off,          % print foldmarks
    numericaldate=off,      % display date in numbers only
	pagenumber=right,	        % Set page numbers from page 2 onwards
	parskip=half,	        % Separates paragraphs with some whitespace,
    headsep=false,          % Seperatorline in the header
    footsepline=true,       % Seperatorline in the footer
			%use parskip=full for more space or comment out to return to default
    foldmarks=off,		    % Prints small fold marks on the left of the page
	]{scrlttr2}

% Set Font: sans serif Latin Modern
\usepackage{lmodern}
\usepackage[T1]{fontenc} % For extra glyphs (accents, etc)
\usepackage{stix} % Use the Stix font by default
\usepackage{lmodern}
\usepackage[utf8]{inputenc}
%\usepackage[english]{babel}
\usepackage[ngerman,english]{babel}
\usepackage{url}
\usepackage{csquotes}
\usepackage{listings}
\usepackage{verbatim}
\usepackage{graphicx}
\usepackage{tabularx}
\renewcommand*{\raggedsignature}{\raggedright} %Signatur rechtsbündig

% Set Page layout:
\usepackage{changepage}
    %\changepage{text height}{text width}{even-side margin}
    %{odd-side margin}{column sep.}
    %{topmargin}{headheight}{headsep}{footskip}
\changepage{}{}{}{}{}{}{}{}{}

%-------------------------------------------------------------------------------
%	YOUR INFORMATION AND LETTER DATE
%-------------------------------------------------------------------------------

\KOMAoptions{foldmarks=true,
foldmarks=false,
fromurl=false,
fromemail=true,
fromphone=false,
fromfax=false,
fromalign=right,
fromrule=off,
footsepline=on,
fromlogo=true,
headsepline = true,
footsepline = true
%fromrule=afteraddress%  % Trennlinie unter dem Briefkopf
}

%\setkomavar{fromlogo}{\parbox[b]{8cm}{\usekomafont{fromaddress}%
%        {\mbox{\LARGE \bfseries Embedded System Design GmbH}}
%        \smallskip}
%}
%

\setkomavar*{urlseparator}{}
\setkomavar{urlseparator}{}
\setkomavar{frombank}{Eine Bank\\BLZ~123\,45\,678\\Kto~123456789}
\setkomavar{fromname}{Marco Israel} % Your name used in the from address
\setkomavar{fromurl}{www.marcois.eu}
\setkomavar{fromaddress}{Am Wickenkamp 38\\ 32351 Stemwede}
\setkomavar{fromemail}{Marco-Israel@online.de}
\setkomavar{signature}{Marco Israel}
\setkomavar{fromphone}{05773 00000000}
\setkomavar{frommobilephone}{0175 00000}
%\setkomavar{place}{Stemwede}
%\includegraphics[⟨options⟩]{⟨file⟩}
\firstfoot{}
\nextfoot{}


%-------------------------------------------------------------------------------
%  HEADER SECTION
%-------------------------------------------------------------------------------
%\firsthead{\centering
%        {\mbox{\LARGE \bfseries Embedded System Design GmbH}}
%}
%

%-------------------------------------------------------------------------------
%  FOOTER SECTION
%-------------------------------------------------------------------------------
%\firstfoot{%
%\centering
%{\renewcommand{\\}{\ {\large\textperiodcentered}\ }
%\small\usekomavar{frombank}
%}%
%}
%




%-------------------------------------------------------------------------------

\begin{document}

%-------------------------------------------------------------------------------
%	ADDRESSEE
%-------------------------------------------------------------------------------

% Addressee name and address
\begin{letter} {Wilhelm Büchner Hochschule \\
Hilpertstr. 31\\
64295 Darmstad}

\setkomavar*{yourref}{AufgabenCode}
\setkomavar{yourref}{FINA-H-XX1-K10}		% Ihr Zeichen
\setkomavar*{yourmail}{HeftKürzel} 	% Ihr Schreiben vom
\setkomavar{yourmail}{FINA-HXX} 	% Ihr Schreiben vom
\setkomavar*{myref}{Auflage}       	% Unser Zeichen
\setkomavar{myref}{0114K10}       	% Unser Zeichen
\setkomavar*{customer}{Matrikel-Nr}
\setkomavar{customer}{580201} 	% Kundennummer
\setkomavar*{invoice}{StudiengangsNr.}   	% Rechnungsnummer
\setkomavar{invoice}{1640}   	% Rechnungsnummer
\setkomavar{place}{Stemwede}		% Ort
\setkomavar{date}{\today}			% Datum

\setkomavar{subject}{Einsendeaufgaben Typ A }

\opening{Sehr geehrte(r) Herr / Frau}

Guten Tag,

im Anhang die Lösungen für o.g. Einsendeaufgabe Typ A,
\\

\begin{itemize}
\item \textbf{Vorab Definition:}
    \begin{itemize}
    \item
        Monatlicher Gewinn nach Fix = Erlöse - Fixkosten
        150.000 - 80.000 - 6.000 =   64.000
    \item Es \textit{soll} = kann = könnte = muss aber nicht eine Spende
        gezahlt werden. Sie ist nicht verpflichtend.
    \item Absetzbarkeit von Spende für Unternehmen ( § 10b Abs. 1 S. 1 EStG):
        \begin{itemize}
        \item 20 Prozent des Gesamtbetrags der Einkünfte oder
        \item 4 Promille der Summe der gesamten Umsätze und der im
            Kalenderjahr aufgewendeten Löhne und Gehälter. Die Zahlung
            kann im Vorfeld dem Finanzamt bekannt gemacht werden um eine
            schnelle Steuererstattung zu sicher zu stellen (Spenden sind
            wohl überlegt und geplant und nicht grundlos).
        \end{itemize}
    \item  Es wird \textit{ angestrebt } = \textbf{grob} geplant/kalkuliert =
        gewünscht; ist aber nicht fixiert - soweit sinnvoll - eine  Liquidität
        zurückzuhalten, wenn möglich.. Dennoch dient die Liquidität um
        Zahlungsspitzen auszugleichen und dient nicht dazu bei null Prozent
        Verzinsung auf dem
        Bankkonto zu ruhen.
    \end{itemize}
\end{itemize}

\begin{itemize}
\item  \textbf{1. Finanzplan Vorab}
\end{itemize}

        \begin{verbatim}
| Monat | Zahlungsfällig nach Fixkosten   | Rest Lequidität                    |
+=======+=================================+====================================+
| 1     | 23.000  +19.000 -8.000 + 4500   | 14.000(- 9500 )  = 0               |
+-------+---------------------------------+------------------------------------+
| 2     | 37.000                          | 27.000 (-500 ) = 26.500            |
+-------+---------------------------------+------------------------------------+
| 3     | 43.000 + 100.000 + 4.500        | -57000 + 10.000 = -47.000          |
+-------+---------------------------------+------------------------------------+
| 4     | 28.000 + 8.000 + 23.000         | -42.000 + 30.000 Steuererstattung  |
|       |                                 | Spenden etwa (30 Prozent) = 12.000 |
+-------+---------------------------------+------------------------------------+
| 5     | 32.000 + 45.000                 | 15.500 ( -10.000 ) = 5.500         |
+-------+---------------------------------+------------------------------------+
| 6     | 56.000                          | = 23.500                           |
+-------+---------------------------------+------------------------------------+
| 7     | 19.000 + 4.500 + 23.000 + 8.000 | = 33.000 (+10.000)                 |
+-------+---------------------------------+------------------------------------+
        \end{verbatim}

Alle Angaben in € (Euro).

\begin{itemize}
    \item \textbf{2. Finanzplan mit Spende und Darlehn}
\end{itemize}

\begin{verbatim}
| Monat | Rest Lequidität    | Aufnahme Darlehn | Zusatz-Zinsen | Neue Mittel |
+=======+====================+==================+===============+=============+
| 1     | = 0                | 0                |               |             |
+-------+--------------------+------------------+---------------+-------------+
| 2     | = 26.500           | 0                |               |             |
+-------+--------------------+------------------+---------------+-------------+
| 3     | = -47.000          | 50.000 (s.o.     | 500           | 2500        |
|       |                    | zur Lequidit#.   |               |             |
+-------+--------------------+------------------+---------------+-------------+
| 4     | = 12.000           | 0                | 500           | 61.500      |
+-------+--------------------+------------------+---------------+-------------+
| 5     | = 5.500            | 0                | 500           | 66500       |
+-------+--------------------+------------------+---------------+-------------+
| 6     | = 23.500           | 0                | 500           | 73.000      |
+-------+--------------------+------------------+---------------+-------------+
| 7     | = 33.000 (+10.000) | 0                | 500           | 82500       |
+-------+--------------------+------------------+---------------+-------------+
\end{verbatim}

Alle Angaben in € (Euro).

\begin{itemize}
    \item \textbf{3. Verhinderung Neukreditaufnahme.}

    \begin{itemize}
        \item S.o Vorab Definition.
        \item Das Darlehn sollte aufgeteilt werden in z.B. zwei Hälften. Die
            empfangene Organisation wird wohl auch in diesem Falle dankbar sein
            und die Steuerliche anrechenbar ist weiterhin im (Geschäfts)Jahr
            gegeben.
        \item Wie in 2 bereits geschehen, sollte dar Liquiditätsbestand in
            Spitzen (März) genutzt werden. Die Darlehenskosten (Zinszahlungen)
            stehen nicht im Verhältnis zu den null Prozent Zinsen auf dem
            Bankkonto (Zinszahlungseingänge gleich null).
    \item Wie definiert ist, soll eine Liquidität vorhanden sein.Gleichzeitig
        ist zwischen den Zeilen, indirekt laut Aufgabenstellung definiert, das
        solche Liquidität nicht auf dem Bankkonto liegen soll, insofern
        anderweitig benötigt (null Prozent Zinsen).
    \end{itemize}
\end{itemize}

\closing{Mit freundlichen Grüßen}
    \includegraphics[scale=0.70]{../unterschrift.png}



%\encl{Myattachments} % Attached documents

%-------------------------------------------------------------------------------
%   Postskriptum
%
%\ps\ PS:\ someText
%-------------------------------------------------------------------------------


%-------------------------------------------------------------------------------
%  Anlagen
%\setkomavar*{enclseparator}{Anlage}
%\encl{%
%  Anlage 1\\
%  Anlage 2%
%}
%-------------------------------------------------------------------------------


%-------------------------------------------------------------------------------
%   Verteiler
%\setkomavar*{ccseparator}{Kopie an}
%\cc{%
%  Verteiler 1\\
%  Verteiler 2%
%}
%-------------------------------------------------------------------------------

\end{letter}

\end{document}

%--- EOF -----------------------------------------------------------------------
%
%
