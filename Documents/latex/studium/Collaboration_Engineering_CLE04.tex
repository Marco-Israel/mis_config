%-------------------------------------------------------------------------------
% @author    Marco Israel (MI)
% @date      2020-12
% @authors   Last changed by:  Marco Israel - 2020.
%
%
% @brief     German DinBrief template
% @details   German DinBrief template
%
%
%  @example
%  "mkpdf -f letter_template.tex"
%
%  @copyright     Available under the MIT License.
%
%			Copyright (C) 2020 Marco Israel (MI).			All rights reserved.
%
% Permission is hereby granted, free of charge, to any person obtaining a copy
% of this software and associated documentation files (the "Software"), to deal
% in
% the Software without restriction, including without limitation the rights to
% use, copy, modify, merge, publish, distribute, sublicense, and/or sell copies
% of the Software, and to permit persons to whom the Software is furnished to
% do so, subject to the following conditions:
%
% The above copyright notice and this permission notice shall and MUST BE
% included in all copies or substantial portions of the Software.
%
% THE SOFTWARE IS PROVIDED "AS IS", WITHOUT WARRANTY OF ANY KIND, EXPRESS OR
% IMPLIED, INCLUDING BUT NOT LIMITED TO THE WARRANTIES OF MERCHANTABILITY,
% FITNESS FOR A PARTICULAR PURPOSE AND NONINFRINGEMENT. IN NO EVENT SHALL THE
% AUTHORS OR COPYRIGHT HOLDERS BE LIABLE FOR ANY CLAIM, DAMAGES OR OTHER
% LIABILITY, WHETHER IN AN ACTION OF CONTRACT, TORT OR OTHERWISE, ARISING FROM,
% OUT OF OR IN CONNECTION WITH THE SOFTWARE OR THE USE OR OTHER DEALINGS IN
% THE SOFTWARE.
%-------------------------------------------------------------------------------

%-------------------------------------------------------------------------------
%	DOCUMENT CONFIGURATIONS
%-------------------------------------------------------------------------------

\documentclass[
    version=last,           % use last release of scrlttr2 package
    DIV=13,                 % ?
    BCOR=0mm,               % ?
    paper=a4,               % paper size
    fontsize=12pt,          % fontsize
    firsthead=on,           % display header on first page
    firstfoot=on,           % display footer on first page
    pagenumber=on,i         % position of the page number
    parskip=half,           % Use indent instead of skip, half, false
    enlargefirstpage=,      % more space on first page
    firsthead=on,           % Display the adress
    fromrule=afteraddress,  % separate the address with a line in letter he
    priority=off,           % print a priority onto the document
    backaddress=true,       % Adress to send back
    refline=dateright,      % location of the date
	fromalign=right,	    % Aligns the from address to the right
    fromemail=on,i          % turn on email of sender
    fromurl=on,             % print URL of sender
    frombank=on,
    fromphone=on,           % turn on phone of sender
    frommobilephone=on      % Mobile Phone number
    fromlogo=on,            % turn on logo of sender
    addrfield=on,           % address field for envelope with window, on or true
    subject=untitled,  % placement of subject, beforeopening or titled
    foldmarks=off,          % print foldmarks
    numericaldate=off,      % display date in numbers only
	pagenumber=right,	        % Set page numbers from page 2 onwards
	parskip=half,	        % Separates paragraphs with some whitespace,
    headsep=false,          % Seperatorline in the header
    footsepline=true,       % Seperatorline in the footer
			%use parskip=full for more space or comment out to return to default
    foldmarks=off,		    % Prints small fold marks on the left of the page
	]{scrlttr2}

% Set Font: sans serif Latin Modern
\usepackage{lmodern}
\usepackage[T1]{fontenc} % For extra glyphs (accents, etc)
\usepackage{stix} % Use the Stix font by default
\usepackage{lmodern}
\usepackage[utf8]{inputenc}
%\usepackage[english]{babel}
\usepackage[ngerman,english]{babel}
\usepackage{url}
\usepackage{csquotes}
\usepackage{listings}
\usepackage{verbatim}
\usepackage{graphicx}
\usepackage{tabularx}
\renewcommand*{\raggedsignature}{\raggedright} %Signatur rechtsbündig

% Set Page layout:
\usepackage{changepage}
    %\changepage{text height}{text width}{even-side margin}
    %{odd-side margin}{column sep.}
    %{topmargin}{headheight}{headsep}{footskip}
\changepage{}{}{}{}{}{}{}{}{}

%-------------------------------------------------------------------------------
%	YOUR INFORMATION AND LETTER DATE
%-------------------------------------------------------------------------------

\KOMAoptions{foldmarks=true,
foldmarks=false,
fromurl=false,
fromemail=true,
fromphone=false,
fromfax=false,
fromalign=right,
fromrule=off,
footsepline=on,
fromlogo=true,
headsepline = true,
footsepline = true
%fromrule=afteraddress%  % Trennlinie unter dem Briefkopf
}

%\setkomavar{fromlogo}{\parbox[b]{8cm}{\usekomafont{fromaddress}%
%        {\mbox{\LARGE \bfseries Embedded System Design GmbH}}
%        \smallskip}
%}
%

\setkomavar*{urlseparator}{}
\setkomavar{urlseparator}{}
\setkomavar{frombank}{Eine Bank\\BLZ~123\,45\,678\\Kto~123456789}
\setkomavar{fromname}{Marco Israel} % Your name used in the from address
\setkomavar{fromurl}{www.marcois.eu}
\setkomavar{fromaddress}{Am Wickenkamp 38\\ 32351 Stemwede}
\setkomavar{fromemail}{Marco-Israel@online.de}
\setkomavar{signature}{Marco Israel}
\setkomavar{fromphone}{05773 00000000}
\setkomavar{frommobilephone}{0175 00000}
%\setkomavar{place}{Stemwede}
%\includegraphics[⟨options⟩]{⟨file⟩}
\firstfoot{}
\nextfoot{}


%-------------------------------------------------------------------------------
%  HEADER SECTION
%-------------------------------------------------------------------------------
%\firsthead{\centering
%        {\mbox{\LARGE \bfseries Embedded System Design GmbH}}
%}
%

%-------------------------------------------------------------------------------
%  FOOTER SECTION
%-------------------------------------------------------------------------------
%\firstfoot{%
%\centering
%{\renewcommand{\\}{\ {\large\textperiodcentered}\ }
%\small\usekomavar{frombank}
%}%
%}
%




%-------------------------------------------------------------------------------

\begin{document}

%-------------------------------------------------------------------------------
%	ADDRESSEE
%-------------------------------------------------------------------------------

% Addressee name and address
\begin{letter} {Wilhelm Büchner Hochschule \\
Hilpertstr. 31\\
64295 Darmstad}

\setkomavar*{yourref}{AufgabenCode}
\setkomavar{yourref}{CLE04-XX1-N01}		% Ihr Zeichen
\setkomavar*{yourmail}{HeftKürzel} 	% Ihr Schreiben vom
\setkomavar{yourmail}{CLE04XX} 	% Ihr Schreiben vom
\setkomavar*{myref}{Auflage}       	% Unser Zeichen
\setkomavar{myref}{CLE04-XX1-N01}
\setkomavar*{customer}{Matrikel-Nr}
\setkomavar{customer}{580201} 	% Kundennummer
\setkomavar*{invoice}{StudiengangsNr.}   	% Rechnungsnummer
\setkomavar{invoice}{1640}   	% Rechnungsnummer
\setkomavar{place}{Stemwede}		% Ort
\setkomavar{date}{\today}			% Datum

\setkomavar{subject}{Einsendeaufgaben Typ A }

\opening{Sehr geehrte(r) Herr / Frau}

Guten Tag,

im Anhang die Lösungen für o.g. Einsendeaufgabe Typ A,
\\
\begin{itemize}
    \item Multiple Choice
        \begin{enumerate}
        \item \textbf{thinkLets 1}: C \\
        \item \textbf{thinkLets 2}: D \\
        \end{enumerate}
            \vspace{1cm}
    \item Textaufgaben
        \begin{enumerate}
        \setcounter{enumi}{2}
        \item \textbf{Elemente des Four-Ways-Framework als Rahmen zur
            Gestaltung der Zusammenarbeit:} \\
            \begin{description}
                \item[Way of thinking]
                    Die Art und Weise des Denkens beinhaltet die dem Ansatz
                    zugrunde liegende Philosophie und damit die Sicht auf die
                    Problemstellung sowie die zugrunde liegenden Annahmen und
                    Fundamente.
                \item [Way of working]
                    Verbindung möglicher Aufgaben, die im Gestaltungsprozess
                    durchgeführt werden. Bezeichnet als  Arbeitsweise.
                 \item [Way of modeling]
                    Modellierung und solche Konzepte zur Darstellung von
                    relevanten Aspekten der Problemstellung
                \item [Way of controlling] Die Managementmethode zur Steuerung
                    bringt die leitenden Aspekte wie Maßregeln und Methoden zur
                    Durchführung eines Gestaltungsansatzes.
            \end{description}
            \vspace{1cm}
            \item \textbf{ Nennen und beschreiben Sie die Elemente eines
                thinkLets:} \\
                ThinkLets sollen wiederholbare, gleiche Ergebnisse-Qualitäten in
                Gruppenprozessen bei gleichem Zeit und Budgetrahmen erzeugt werden.
                ThinkLet sind dabei Bausteine die Teilprozesse repräsentieren
                und zu einem Prozess zusammen gesetzt werden können.
                Das anfängliche Konzept eines ThinkLets beschreibt nachfolgende
                drei Elementare Komponenten:
                \begin{description}
                    \item[Das Werkzeug: ] Die Technologie, welche das Muster der
                        Zusammenarbeit erzeugt. Von Zetteln, Stiften und
                        Flipcharts bis hin zu Hardware- und Softwaresystemen.
                    \item [Die Konfiguration: ] Beschreibt, wie das
                        \textit{Werkzeug} gebrauchsfertig gemacht
                        (eingerichtet) wird. Z.B. wie werden anonym Vorschläge
                        in das / mit dem Werkzeug eingebracht. Vorbereitende
                        Aufgaben der Teilnehmer / der Gruppenmitglieder
                    \item [ Das Skript: ] Dieses beschreibt alle Aussagen und
                        Handlungen, welche der \textit{Facilitator} tun/sagen
                        sollte, bzw. wie er mit der Gruppe,den Werkzeugen, dem
                        Input und den Ergebnissen (Output) umgehen
                        umgehen soll (im Rahmen eines ThinkLets).
                \end{description}
            \vspace{1cm}
            \item \textbf{ Fünf ThinkLets und representative ihre Muster der
                Zusammenarbeit: }
                \begin{description}
                    \item[ FreeBrainstorming: ]  \hfill
                        \begin{itemize}
                        \item \textit{Generierung:} Durch das ThinkLet werden
                                Gruppenmitglieder aus gewohnten Denkmustern
                                herausgeholt und Ihre Energie in Richtung neuer
                                Ideen gebracht.
                            \item \textit{Verdeutlichung:} Die Ideen anderer
                                        können aufgegriffen und ergänzt werden.

                                    \item (ggf. \textit{ Reduktion: } Das ThinkLet
                                        verhindert eine Informationsüberflutung da
                                        Ideen anderer nicht neu erfunden werden
                                        müssen sonder aufgegriffen werden können.)
                        \end{itemize}
                    \item [ OnePage, DealersChoice] \hfill
                        \begin{itemize}
                        \item \textit{Generierung} und \textit{Verdeutlichung}:
                            wie oben in FreeBrainstorming
                        \end{itemize}
                    \item [ LeafHopper: ] \hfill
                        \begin{itemize}
                        \item \textit{Generierung} und \textit{Verdeutlichung}:
                            Eine durch Diskussionsthemen gegliederte
                            Kommentar-Sammlung.
                        \end{itemize}
                    \item [ FastFocus, Conentration, (PintheTailontheDonkey)] \hfill
                            \begin{itemize}
                                \item \textit{Verdeutlichung}:
                                    Durch die Anwendung dieses ThinkLets wird
                                    sich die Gruppe mehr über die Bedeutungen
                                    und Formulierung der festgehaltenen
                                    Ergebnissen einig.
                                \item \textit{Gliederung und Organeisirrung}:
                                    Das Ergebnis dieses ThinkLets ist eine
                                    ordentliche, nicht redundante Liste der
                                    Ergebnisse einer Brainstorming-Aktivität
                                    (thinkLet).
                                \item \textit{Reduzierung:} Durch das
                                        Kathegorisieren, Gliedern und
                                        Organiseren werden gleichzeitig die
                                        (teils rohen) Brainstorming Ergebnisse
                                        neu betrachtet und zum teil Redundanzen
                                        aussortiert.
                            \end{itemize}
                        \item [ ThemeSeeker, PopcornSort, ChauffeurSort, Evolution] \hfill
                            \begin{itemize}
                            \item \textit{Gliederung und Organeisirrung}:
                                Diese ThinkLets erzeugt als Output eine
                                Zusammenstellung von Kategorien oder bekommen
                                sie direkt als Input, um mit diesen die Inhalte
                                eines Brainstorming im Anschluss zu gliedern und
                                zusammen zu fassen.
                            \item (\textit{Verdeutlichung}: Durch die
                                Kategorisierung und das erarbeiten von
                                Kategorien im vorfeld, findet auch
                                eine Verdeutlichung der Brainstorming Ergebnisse
                                statt. Denn nur wenn diese Ergebnisse verstanden
                                worden sind, können richtige Kategorien
                                gebildet werden und die Ergebnisse richtig
                                zugeordnet werden.)
                            \end{itemize}
                        \item [ StrawPoll: ] \hfill
                            \begin{itemize}
                            \item \textit{Verdeutlichung:} Das ThinkLet überprüft den
                                Konsens einer Gruppe und deckt gleich oder
                                unterschiedliche Ansichten über Ergebnisse auf.
                            \item \textit{Bewertung, Einigung und Konsensbildung}:
                                Durch ausgewählte Methoden und Kriterien der
                                Abstimmung werden die Brainstorming Ergebnisse
                                (und Ihre Kategorien) durch die Teilnehmer
                                bewertet und eine Liste mit gegliederte
                                und bewerteten Positionen entsteht. Je nach
                                Abstimmungskriterien und Abstimmungsmethoden
                                wird hierdurch auch eine Einigung und
                                Konsensbildung erreicht und eine grafische
                                Strukturdarstellung des Konsens innerhalb der Gruppe
                                kann erzeugt werden.
                            \end{itemize}
                            \item [ MoodRing: ] \hfill
                                \begin{itemize}
                                    \item \textit{Verdeutlichung, Einigung,
                                        Konsensbildung}: Durch dieses thinkLet
                                        werden bisherige Ergebnisse diskutiert,
                                        Unklarheiten und Fehler korrigiert und
                                        gleichzeit das Verständnis über die
                                        Ergebnisse weiter verdeutlicht.
                                \end{itemize}
                \end{description}

            \vspace{1cm}
        \item \textbf{Die Bedeutung von Übergängen in thinkLets: } \\
            Da thinkLets jeweils einzelne Teile eines gesamten Prozesses
            darstellen und gleichzeitig unabhängig voneinander sind sind und
            nahezu beliebig kombiniert
            werden können, müssen Übergänge zwischen den einzelnen
            Arbeitsschritten / Prozessteilen und damit zwischen den thinkLets,
            geschaffen werden. So benötigt jedes thinkLet bestimmte
            Rahmenbedingungen um funktionieren zu können. Etwa den Input an
            Daten, die genutzten Werkzeuge oder der ein Arbeitsort und
            Verantwortlichkeiten. Die Übergänge definieren dabei alle
            (Weiter-)Entwicklungen, Maßnahmen und Aktivitäten, die erfolgen
            müssen, um die Teilnehmer vom Ende eines thinkLets zum Beginn des
            nächsten thinkLets (Prozessschrittes) zu begleiten. Aspekte
            die zur Ausgestaltung eins Überganges berücksichtigt werden
            müssen, sind nachfolgende:
            \begin{description}
                \item [ Changes of technology: ]
                    Ggf. muss die Technologie bzw. das Werkzeug zwischen
                    verschiedenen thinkLets getauscht werden
                \item [ Changes of data: ]
                    Ggf. muss der Output des zuletzt angewendeten thinkLets so
                    angepasst werden, das es dem ausgewählten nachfolgenden
                    thinkLet genügt.
                \item [ Changes of orientation: ]
                    Der Gruppe muss signalisiert werden, wann eine Aktivität /
                    ein thinkLet endet und ein neues beginnt. Damit einhergehend
                    müssen zum Ende einer Aktivität die Ergebnisse reflektiert
                    oder zusammengefasst werden.
                \item [ Changes of location: ]
                    Ebenso wie die Technologie / das Werkzeug,
                    kann es auch notwendig sein, den Arbeitsort zu wechseln
                    (ggf. sogar in Abhängigkeit der Technologie / des Werkzeugess).
                \item [ Changes of membership ] Genau wie die Technologie oder
                    der Arbeitsort, kann oder muss es auch notwendig sein, die
                    Zusammenstellung des Teams / der Teams zu verändern:
                    beispielsweise durch austauschen verschiedener Personen
                    und/oder der Positionen.
            \end{description}


        \end{enumerate}
\end{itemize}
\vspace{1cm}
Mit freundlichen Grüßen \\
\\
    \includegraphics[scale=0.70]{../unterschrift.png}
    \\
    Marco Israel



%\encl{Myattachments} % Attached documents

%-------------------------------------------------------------------------------
%   Postskriptum
%
%\ps\ PS:\ someText
%-------------------------------------------------------------------------------


%-------------------------------------------------------------------------------
%  Anlagen
%\setkomavar*{enclseparator}{Anlage}
%\encl{%
%  Anlage 1\\
%  Anlage 2%
%}
%-------------------------------------------------------------------------------


%-------------------------------------------------------------------------------
%   Verteiler
%\setkomavar*{ccseparator}{Kopie an}
%\cc{%
%  Verteiler 1\\
%  Verteiler 2%
%}
%-------------------------------------------------------------------------------

\end{letter}

\end{document}

%--- EOF -----------------------------------------------------------------------
%
%
