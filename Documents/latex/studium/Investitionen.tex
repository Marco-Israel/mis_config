%-------------------------------------------------------------------------------
% @author    Marco Israel (MI)
% @date      2020-12
% @authors   Last changed by:  Marco Israel - 2020.
%
%
% @brief     German DinBrief template
% @details   German DinBrief template
%
%
%  @example
%  "mkpdf -f letter_template.tex"
%
%  @copyright     Available under the MIT License.
%
%			Copyright (C) 2020 Marco Israel (MI).			All rights reserved.
%
% Permission is hereby granted, free of charge, to any person obtaining a copy
% of this software and associated documentation files (the "Software"), to deal
% in
% the Software without restriction, including without limitation the rights to
% use, copy, modify, merge, publish, distribute, sublicense, and/or sell copies
% of the Software, and to permit persons to whom the Software is furnished to
% do so, subject to the following conditions:
%
% The above copyright notice and this permission notice shall and MUST BE
% included in all copies or substantial portions of the Software.
%
% THE SOFTWARE IS PROVIDED "AS IS", WITHOUT WARRANTY OF ANY KIND, EXPRESS OR
% IMPLIED, INCLUDING BUT NOT LIMITED TO THE WARRANTIES OF MERCHANTABILITY,
% FITNESS FOR A PARTICULAR PURPOSE AND NONINFRINGEMENT. IN NO EVENT SHALL THE
% AUTHORS OR COPYRIGHT HOLDERS BE LIABLE FOR ANY CLAIM, DAMAGES OR OTHER
% LIABILITY, WHETHER IN AN ACTION OF CONTRACT, TORT OR OTHERWISE, ARISING FROM,
% OUT OF OR IN CONNECTION WITH THE SOFTWARE OR THE USE OR OTHER DEALINGS IN
% THE SOFTWARE.
%-------------------------------------------------------------------------------

%-------------------------------------------------------------------------------
%	DOCUMENT CONFIGURATIONS
%-------------------------------------------------------------------------------

\documentclass[
    version=last,           % use last release of scrlttr2 package
    DIV=13,                 % ?
    BCOR=0mm,               % ?
    paper=a4,               % paper size
    fontsize=12pt,          % fontsize
    firsthead=on,           % display header on first page
    firstfoot=on,           % display footer on first page
    pagenumber=on,i         % position of the page number
    parskip=half,           % Use indent instead of skip, half, false
    enlargefirstpage=,      % more space on first page
    firsthead=on,           % Display the adress
    fromrule=afteraddress,  % separate the address with a line in letter he
    priority=off,           % print a priority onto the document
    backaddress=true,       % Adress to send back
    refline=dateright,      % location of the date
	fromalign=right,	    % Aligns the from address to the right
    fromemail=on,i          % turn on email of sender
    fromurl=on,             % print URL of sender
    frombank=on,
    fromphone=on,           % turn on phone of sender
    frommobilephone=on      % Mobile Phone number
    fromlogo=on,            % turn on logo of sender
    addrfield=on,           % address field for envelope with window, on or true
    subject=untitled,  % placement of subject, beforeopening or titled
    foldmarks=off,          % print foldmarks
    numericaldate=off,      % display date in numbers only
	pagenumber=right,	        % Set page numbers from page 2 onwards
	parskip=half,	        % Separates paragraphs with some whitespace,
    headsep=false,          % Seperatorline in the header
    footsepline=true,       % Seperatorline in the footer
			%use parskip=full for more space or comment out to return to default
    foldmarks=off,		    % Prints small fold marks on the left of the page
	]{scrlttr2}

% Set Font: sans serif Latin Modern
\usepackage{lmodern}
\usepackage[T1]{fontenc} % For extra glyphs (accents, etc)
\usepackage{stix} % Use the Stix font by default
\usepackage{lmodern}
\usepackage[utf8]{inputenc}
%\usepackage[english]{babel}
\usepackage[ngerman,english]{babel}
\usepackage{url}
\usepackage{csquotes}
\usepackage{listings}
\usepackage{verbatim}
\usepackage{graphicx}
\usepackage{tabularx}
\renewcommand*{\raggedsignature}{\raggedright} %Signatur rechtsbündig

% Set Page layout:
\usepackage{changepage}
    %\changepage{text height}{text width}{even-side margin}
    %{odd-side margin}{column sep.}
    %{topmargin}{headheight}{headsep}{footskip}
\changepage{}{}{}{}{}{}{}{}{}

%-------------------------------------------------------------------------------
%	YOUR INFORMATION AND LETTER DATE
%-------------------------------------------------------------------------------

\KOMAoptions{foldmarks=true,
foldmarks=false,
fromurl=false,
fromemail=true,
fromphone=false,
fromfax=false,
fromalign=right,
fromrule=off,
footsepline=on,
fromlogo=true,
headsepline = true,
footsepline = true
%fromrule=afteraddress%  % Trennlinie unter dem Briefkopf
}

%\setkomavar{fromlogo}{\parbox[b]{8cm}{\usekomafont{fromaddress}%
%        {\mbox{\LARGE \bfseries Embedded System Design GmbH}}
%        \smallskip}
%}
%

\setkomavar*{urlseparator}{}
\setkomavar{urlseparator}{}
\setkomavar{frombank}{Eine Bank\\BLZ~123\,45\,678\\Kto~123456789}
\setkomavar{fromname}{Marco Israel} % Your name used in the from address
\setkomavar{fromurl}{www.marcois.eu}
\setkomavar{fromaddress}{Am Wickenkamp 38\\ 32351 Stemwede}
\setkomavar{fromemail}{Marco-Israel@online.de}
\setkomavar{signature}{Marco Israel}
\setkomavar{fromphone}{05773 00000000}
\setkomavar{frommobilephone}{0175 00000}
%\setkomavar{place}{Stemwede}
%\includegraphics[⟨options⟩]{⟨file⟩}
\firstfoot{}
\nextfoot{}


%-------------------------------------------------------------------------------
%  HEADER SECTION
%-------------------------------------------------------------------------------
%\firsthead{\centering
%        {\mbox{\LARGE \bfseries Embedded System Design GmbH}}
%}
%

%-------------------------------------------------------------------------------
%  FOOTER SECTION
%-------------------------------------------------------------------------------
%\firstfoot{%
%\centering
%{\renewcommand{\\}{\ {\large\textperiodcentered}\ }
%\small\usekomavar{frombank}
%}%
%}
%




%-------------------------------------------------------------------------------

\begin{document}

%-------------------------------------------------------------------------------
%	ADDRESSEE
%-------------------------------------------------------------------------------

% Addressee name and address
\begin{letter} {Wilhelm Büchner Hochschule \\
Hilpertstr. 31\\
64295 Darmstad}

\setkomavar*{yourref}{AufgabenCode}
\setkomavar*{yourmail}{INVE-H-XX1-K06} 	% Ihr Schreiben vom
\setkomavar{yourmail}{INV-HXX} 	% Ihr Schreiben vom
\setkomavar*{myref}{Auflage}       	% Unser Zeichen
\setkomavar{myref}{0309 K06}       	% Unser Zeichen
\setkomavar*{customer}{Matrikel-Nr}
\setkomavar{customer}{580201} 	% Kundennummer
\setkomavar*{invoice}{StudiengangsNr.}   	% Rechnungsnummer
\setkomavar{invoice}{1640}   	% Rechnungsnummer
\setkomavar{place}{Stemwede}		% Ort
\setkomavar{date}{\today}			% Datum

\setkomavar{subject}{Einsendeaufgaben Typ A }

\opening{Sehr geehrte(r) Herr / Frau}
Die Lösungen für o.g. Einsendeaufgabe Typ A:


\begin{enumerate}
    \item \textbf{Kostenvergleichsrechnung:} Kostengünstigster Maschine für eine Möbelfabrik von zwei.

\begin{tabular}{| c | c | c | }
\hline & \textbf{Maschine 1} & \textbf{Maschine 2} \\
\hline Fixkosten je Periode & 33.000 & 66.000 \\
\hline Variable Periodenkosten & 572.000 & 280000 \\
\hline Periodenkosten Gesamt & 605.000 & 346.000 \\
\hline Fixkosten je Stück & 3 & 6,6 \\
\hline Variable Stückkosten & 50 & 28 \\
\hline Stückkosten gesamt & 55 & 34,6 \\
\hline
\end{tabular}

Die Maschine 2 ist trotz höherer Anschaffungskosten und geringerer Stückzahl
sowohl in den Gesamtkosten, als auch in den Stückkosten günstiger.
Die Frage bleibt offen, ob der Differenzbetrag anderweitig eingesetzt werden
könnte, das in Kombination mit Maschine 1 das gesamtpaket eine höhere
Rentabilität in Summe hat, gegenüber Maschine 2.
\item \textbf{Gewinnvergleichsrechnung} aus 1. mit verkaufspreis von 80 je
    Stück.

   \begin{tabular}{|c|c|c| }
   \hline & Maschine 1 & 	 Maschine 2\\
   \hline 80    &   880.000 & 800.000 \\
   \hline Periodenkosten & 605.000 & 346.000 \\
   \hline Gewinn & 275.000 & 454.000 \\
   \hline
   \end{tabular}

Den Differenzbetrag unbeachtet gelassen, erzielt Maschine 2 auch in der
Gewinnvergleichsrechnung einen höheren Periodenerfolg.

\item \textbf{Rentabilitätsrechnung im Durchschnitt und in Prognose}

    \begin{tabular}{|c|c|c| }
    \hline  & I & 	 II \\
    \hline Durchschnitt &  14,4 \% & 8,6 \% \\
    \hline Prognose & 10 \% & 6,6 \% \\
    \hline
    \end{tabular}

\item \textbf{ Rangfolge verschiedener Infektionsobjekte für 400T€}

    \begin{enumerate}
        \item ~408T€
        \item ~360T€
        \item ~287T€ >
        \item ~287T€ <
        \item ~129T€
    \end{enumerate}

\item \textbf{ Projektbeteiligung A oder B: Größter Return on Investment (ROI) }
    \begin{description}
    \item[ROI A:]  2.000.000 / 700.000 = 2,86 = 2 Jahre, 10 Monate
    \item[ROI B:] 1.500.000 / 600.000 = 2,5 = 2 Jahre, 6 Monate
    \end{description}

    Eine Beteiligung an Projekt A hat einen schnelleren ROI und ist damit zu
    bevorzugen..
\item \textbf{ Amortisierung-Zeitpunkt einer Plotter-Neuanschaffung }

    Die Ingestion würde sich nach 3 Jahre, 1 Monaten amortisieren: \\

    1750T€/4 = 437,5T€ Abschreibung \\
    (437,5 + (120T€ * 1,1 )) * 3 Jahre = 1708,5T€ \\
    Rest Jahr 4: 41,5 T€ \\
    (437,5 + (120 * 1,1\^3)) / 12 Monate = 61,2T€ je Monat
\end{enumerate}

\closing{Mit freundlichen Grüßen}
    \includegraphics[scale=0.70]{../unterschrift.png}



%\encl{Myattachments} % Attached documents

%-------------------------------------------------------------------------------
%   Postskriptum
%
%\ps\ PS:\ someText
%-------------------------------------------------------------------------------


%-------------------------------------------------------------------------------
%  Anlagen
%\setkomavar*{enclseparator}{Anlage}
%\encl{%
%  Anlage 1\\
%  Anlage 2%
%}
%-------------------------------------------------------------------------------


%-------------------------------------------------------------------------------
%   Verteiler
%\setkomavar*{ccseparator}{Kopie an}
%\cc{%
%  Verteiler 1\\
%  Verteiler 2%
%}
%-------------------------------------------------------------------------------

\end{letter}

\end{document}

%--- EOF -----------------------------------------------------------------------
%
%
