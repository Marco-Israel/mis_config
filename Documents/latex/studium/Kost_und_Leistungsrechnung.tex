%-------------------------------------------------------------------------------
% @author    Marco Israel (MI)
% @date      2020-12
% @authors   Last changed by:  Marco Israel - 2020.
%
%
% @brief     German DinBrief template
% @details   German DinBrief template
%
%
%  @example
%  "mkpdf -f letter_template.tex"
%
%  @copyright     Available under the MIT License.
%
%			Copyright (C) 2020 Marco Israel (MI).			All rights reserved.
%
% Permission is hereby granted, free of charge, to any person obtaining a copy
% of this software and associated documentation files (the "Software"), to deal
% in
% the Software without restriction, including without limitation the rights to
% use, copy, modify, merge, publish, distribute, sublicense, and/or sell copies
% of the Software, and to permit persons to whom the Software is furnished to
% do so, subject to the following conditions:
%
% The above copyright notice and this permission notice shall and MUST BE
% included in all copies or substantial portions of the Software.
%
% THE SOFTWARE IS PROVIDED "AS IS", WITHOUT WARRANTY OF ANY KIND, EXPRESS OR
% IMPLIED, INCLUDING BUT NOT LIMITED TO THE WARRANTIES OF MERCHANTABILITY,
% FITNESS FOR A PARTICULAR PURPOSE AND NONINFRINGEMENT. IN NO EVENT SHALL THE
% AUTHORS OR COPYRIGHT HOLDERS BE LIABLE FOR ANY CLAIM, DAMAGES OR OTHER
% LIABILITY, WHETHER IN AN ACTION OF CONTRACT, TORT OR OTHERWISE, ARISING FROM,
% OUT OF OR IN CONNECTION WITH THE SOFTWARE OR THE USE OR OTHER DEALINGS IN
% THE SOFTWARE.
%-------------------------------------------------------------------------------

%-------------------------------------------------------------------------------
%	DOCUMENT CONFIGURATIONS
%-------------------------------------------------------------------------------

\documentclass[
    version=last,           % use last release of scrlttr2 package
    DIV=13,                 % ?
    BCOR=0mm,               % ?
    paper=a4,               % paper size
    fontsize=12pt,          % fontsize
    firsthead=on,           % display header on first page
    firstfoot=on,           % display footer on first page
    pagenumber=on,i         % position of the page number
    parskip=half,           % Use indent instead of skip, half, false
    enlargefirstpage=,      % more space on first page
    firsthead=on,           % Display the adress
    fromrule=afteraddress,  % separate the address with a line in letter he
    priority=off,           % print a priority onto the document
    backaddress=true,       % Adress to send back
    refline=dateright,      % location of the date
	fromalign=right,	    % Aligns the from address to the right
    fromemail=on,i          % turn on email of sender
    fromurl=on,             % print URL of sender
    frombank=on,
    fromphone=on,           % turn on phone of sender
    frommobilephone=on      % Mobile Phone number
    fromlogo=on,            % turn on logo of sender
    addrfield=on,           % address field for envelope with window, on or true
    subject=untitled,  % placement of subject, beforeopening or titled
    foldmarks=off,          % print foldmarks
    numericaldate=off,      % display date in numbers only
	pagenumber=right,	        % Set page numbers from page 2 onwards
	parskip=half,	        % Separates paragraphs with some whitespace,
    headsep=false,          % Seperatorline in the header
    footsepline=true,       % Seperatorline in the footer
			%use parskip=full for more space or comment out to return to default
    foldmarks=off,		    % Prints small fold marks on the left of the page
	]{scrlttr2}

% Set Font: sans serif Latin Modern
\usepackage{lmodern}
\usepackage[T1]{fontenc} % For extra glyphs (accents, etc)
\usepackage{stix} % Use the Stix font by default
\usepackage{lmodern}
\usepackage[utf8]{inputenc}
%\usepackage[english]{babel}
\usepackage[ngerman,english]{babel}
\usepackage{url}
\usepackage{csquotes}
\usepackage{listings}
\usepackage{verbatim}
\usepackage{graphicx}
\usepackage{tabularx}
\renewcommand*{\raggedsignature}{\raggedright} %Signatur rechtsbündig

% Set Page layout:
\usepackage{changepage}
    %\changepage{text height}{text width}{even-side margin}
    %{odd-side margin}{column sep.}
    %{topmargin}{headheight}{headsep}{footskip}
\changepage{}{}{}{}{}{}{}{}{}

%-------------------------------------------------------------------------------
%	YOUR INFORMATION AND LETTER DATE
%-------------------------------------------------------------------------------

\KOMAoptions{foldmarks=true,
foldmarks=false,
fromurl=false,
fromemail=true,
fromphone=false,
fromfax=false,
fromalign=right,
fromrule=off,
footsepline=on,
fromlogo=true,
headsepline = true,
footsepline = true
%fromrule=afteraddress%  % Trennlinie unter dem Briefkopf
}

\setkomavar{fromlogo}{\parbox[b]{8cm}{\usekomafont{fromaddress}%
%        {\mbox{\LARGE \bfseries Embedded System Design GmbH}}
%        \smallskip}
%}
%

\setkomavar*{urlseparator}{}
\setkomavar{urlseparator}{}
\setkomavar{frombank}{Eine Bank\\BLZ~123\,45\,678\\Kto~123456789}
\setkomavar{fromname}{Marco Israel} % Your name used in the from address
\setkomavar{fromurl}{www.marcois.eu}
\setkomavar{fromaddress}{Am Wickenkamp 38\\ 32351 Stemwede}
\setkomavar{fromemail}{Marco-Israel@online.de}
\setkomavar{signature}{Marco Israel}
\setkomavar{fromphone}{05773 00000000}
\setkomavar{frommobilephone}{0175 00000}
%\setkomavar{place}{Stemwede}
%\includegraphics[⟨options⟩]{⟨file⟩}
\firstfoot{}
\nextfoot{}


%-------------------------------------------------------------------------------
%  HEADER SECTION
%-------------------------------------------------------------------------------
%\firsthead{\centering
%        {\mbox{\LARGE \bfseries Embedded System Design GmbH}}
%}
%

%-------------------------------------------------------------------------------
%  FOOTER SECTION
%-------------------------------------------------------------------------------
%\firstfoot{%
%\centering
%{\renewcommand{\\}{\ {\large\textperiodcentered}\ }
%\small\usekomavar{frombank}
%}%
%}
%




%-------------------------------------------------------------------------------

\begin{document}

%-------------------------------------------------------------------------------
%	ADDRESSEE
%-------------------------------------------------------------------------------

% Addressee name and address
\begin{letter} {Wilhelm Büchner Hochschule \\
Hilpertstr. 31\\
64295 Darmstad}

\setkomavar*{yourref}{AufgabenCode}
\setkomavar{yourref}{KOST-H-XX1-K08}		% Ihr Zeichen
\setkomavar*{yourmail}{HeftKürzel} 	% Ihr Schreiben vom
\setkomavar{yourmail}{KOST-H} 	% Ihr Schreiben vom
\setkomavar*{myref}{DruckCode}       	% Unser Zeichen
\setkomavar{myref}{0114K08}       	% Unser Zeichen
\setkomavar*{customer}{Matrikel-Nr}
\setkomavar{customer}{580201} 	% Kundennummer
\setkomavar*{invoice}{StudiengangsNr.}   	% Rechnungsnummer
\setkomavar{invoice}{1640}   	% Rechnungsnummer
\setkomavar{place}{Stemwede}		% Ort
\setkomavar{date}{\today}			% Datum

\setkomavar{subject}{Einsendeaufgaben Typ A }

\opening{Sehr geehrte(r) Herr / Frau}

Guten Tag,

im Anhang die Lösungen für o.g. Einsendeaufgabe Typ A,

\begin{enumerate}
    \item Nicht zufriedenstellendes Betriebsergebnis der EXIM \underline{KG}
        \begin{enumerate}
            \item Produktivität und Wirtschaftlichkeit
                \begin{itemize}
                    \item 8.000 kg Holz / 250 Tische = 32kb Holz je Tisch.
                    \item 8.000 kg Holz * 50€ = 400.000 Euro Wareneinsatz / 250
                        Tische = 1600 Euro Wareneinsatz je Tisch.
                \end{itemize}
            \item Verbesserungspotential:
                Da es bei dieser Frage kein \textit{Richti} oder \textit{Falsch}
                gibt, versuche ich es einfach mal; wie in einem Supermark darf
                man mein \textit{Angebot kaufen}, oder es für andere
                \textit{liegen lassen}:
                \begin{itemize}
                    \item VK-Preis 6.000 Euro ./. 1.600 Euro Wareneinsatz je
                        Tisch = 5.400 Euro Gewinn vor Steuern, je Tisch.
                    \item Das ist eine Gewinnmarge bzw, ein Aufschlag von 375
                        Prozent.
                    \item Da keine andern Kosten angegeben --> entstanden sind und die
                        Fechtform eine \textit{KG} ist, "folgte direkt"
                        [Mathematikerblogik, siehe Stochastik) das es sich um ein
                        Einzelunternehmen ohne Angestellten und weiter Kosten handelt. Bei
                        einer Gewinnmarge von 5400 Euro je Tisch und 250 Tischen im
                        Quartal (3 Monate) Beträgt der Gewinn 366.666,67 Euro \textbf{je
                        Monat}. Das ist bei einer Personengesellschaft ein Steuersatz von
                        45 Prozent zzgl. 2 Prozent Solidaritätszuschlag = 47 Prozent +
                        19 Prozent auf Konsumgüter die als Unternehmen abziehbar
                        wären wie auch  nicht Abziehbarer Unternehmerlohn-Lohn = etwa
                        50 Steuerbelastung auf diese Einkunft (+ optional 8-9 Prozent
                        Kirchensteuer) --> Das aktuell größte Problem dieses
                        Solo-Selbstständigen sind die Steuerlichen Abgaben. Durch
                        Gründung einer Rechtsform GmbH lassen sich die Kosten auf
                        \textbf{gesamt und fix} 25 Prozent bis 35 Prozent (anhängig
                        des U-Sitz und der Vertriebs- und Produktionsstätten). Die
                        Gründungskosten und Verwaltungskosten hätte er in nur einem
                        Monat durch Steuervorteil bereits gespart.
                    \item Sollten andere Lösungswege gewünscht sein, lautet
                        meine weitere Empfehlung: Geben Sie mir doch mehr Zahlen
                        als Rechengrundlage. :)
                \end{itemize}

            \item Schwankende Produktivität vs Wirtschaftlichkeit
                \begin{itemize}
                    \item Ja, z.B. bei Preisschwankungen am Markt wie veränderte
                        Warenkosten oder \newline schwankende Verkaufpreise
                        (Angebot / Nachfrage). Oder etwa Änderungen in der
                        Gesetzgebung.
                \end{itemize}

        \end{enumerate}

    \item Zahlen und Daten der Velo GmbH
        \begin{itemize}
            \item 500.000 Euro + 100.000 Euro / (400 Euro VK Preis - 150
                Euro Kosten je Stück) = 2400 Stück für Break Even Point
            \item 100.000 Euro / 2400 Euro = 42 Euro Gewinn je Stück.
            \item Klitsche Menge = 500.000 Euro / 250 Euro je Stück =
                2.000 Stück.
        \end{itemize}
    \item Überprüfung des Zuschlages der Yellow KG.
        \begin{enumerate}
            \item Berechnung der Selbstkosten \newline
            \begin{tabular}{ |l|l| }
                    \hline Material-Einzelkosen & 500 Euro \\
                    \hline Material-Gemeinkosten 8 Prozent & 40 Euro \\
                    \hline Fertigungs-Einzelkosten & 900 Euro \\
                    \hline Fertigungs-Gemeinkosten 13 Prozent & 117 Euro \\
                    \hline \textbf{Herstellungskosten} & \textbf{1557 Euro} \\
                    \hline Verwaltungsgemeinkosten 4 Prozent & 62,28 Euro \\
                    \hline Vertriebsgemeinkosten 5 Prozent &  77,85 Euro \\
                    \hline \textbf{Selbstkosten} & \textbf{1.697,13 Euro} \\
                    \hline
            \end{tabular}

                    Nicht Teil der Frage, aber aus meine Sicht sind die
                    Zuschläge in Summe mindestens 5 Prozent zu hoch. Leider
                    Lasssen sich 250.000 zu 500 nicht ohne waters im Verhältnis
                    setzten zu 400.000 zu 900; jeweils Euro.

                \item Position Abschreibungen für Regale:
                    \begin{itemize}
                        \item Fertigungsgemeinkosten, da die Lagerregal benötigt werden
                            um Rohstoffe zwischen zu lagern.
                    \end{itemize}
                \item Position Kosten Telefonanlage
                    \begin{itemize}
                        \item  Verwaltungsgemeinkosten
                    \end{itemize}

        \end{enumerate}
    \item Sollte das (eine) Produkt mit dem schlechtesten Ergebnis aus der
        Produktion genommen werden?
        \begin{itemize}
            \item Vorerst (kurzfristig) Nein, erst
                wenn ein 1. Ein Nachfolgeprodukt mit kurz- bis mittelfristig
                besserer bis weitaus besserer Gewinnmarge produktionsseereit
                ist, oder die Produktionsanlagen durch die anderen Produkte
                immer noch ausgelastet werden können. Ankerfalls generiert
                selbst das schlechteste Produkt noch einen kleinen Gewinn; Trägt
                jedoch im vollen Umfang mit zum Deckungsbeitrag bei. Würde man
                das Produkt aus dem Sortiment genommen werden, müssten die
                verbleibenden Produkte einen höheren Deckungsbeitag leisten um
                Fixkosten zu begleichen. Dieses würde den Nettogewinn vor
                Steuern dieser Produkte reduzieren.
        \end{itemize}
\end{enumerate}

\closing{Mit freundlichen Grüßen}
    \includegraphics[scale=0.1]{/data/50_Weiterbildung/Studium/MBA/Unterschrift.jpg}


%\encl{Myattachments} % Attached documents

%-------------------------------------------------------------------------------
%   Postskriptum
%
%\ps\ PS:\ someText
%-------------------------------------------------------------------------------


%-------------------------------------------------------------------------------
%  Anlagen
%\setkomavar*{enclseparator}{Anlage}
%\encl{%
%  Anlage 1\\
%  Anlage 2%
%}
%-------------------------------------------------------------------------------


%-------------------------------------------------------------------------------
%   Verteiler
%\setkomavar*{ccseparator}{Kopie an}
%\cc{%
%  Verteiler 1\\
%  Verteiler 2%
%}
%-------------------------------------------------------------------------------

\end{letter}

\end{document}

%--- EOF -----------------------------------------------------------------------
%
%
%
