%-------------------------------------------------------------------------------
% @author    Marco Israel (MI)
% @date      2020-12
% @authors   Last changed by:  Marco Israel - 2020.
%
%
% @brief     German DinBrief template
% @details   German DinBrief template
%
%
%  @example
%  "mkpdf -f letter_template.tex"
%
%  @copyright     Available under the MIT License.
%
%			Copyright (C) 2020 Marco Israel (MI).			All rights reserved.
%
% Permission is hereby granted, free of charge, to any person obtaining a copy
% of this software and associated documentation files (the "Software"), to deal
% in
% the Software without restriction, including without limitation the rights to
% use, copy, modify, merge, publish, distribute, sublicense, and/or sell copies
% of the Software, and to permit persons to whom the Software is furnished to
% do so, subject to the following conditions:
%
% The above copyright notice and this permission notice shall and MUST BE
% included in all copies or substantial portions of the Software.
%
% THE SOFTWARE IS PROVIDED "AS IS", WITHOUT WARRANTY OF ANY KIND, EXPRESS OR
% IMPLIED, INCLUDING BUT NOT LIMITED TO THE WARRANTIES OF MERCHANTABILITY,
% FITNESS FOR A PARTICULAR PURPOSE AND NONINFRINGEMENT. IN NO EVENT SHALL THE
% AUTHORS OR COPYRIGHT HOLDERS BE LIABLE FOR ANY CLAIM, DAMAGES OR OTHER
% LIABILITY, WHETHER IN AN ACTION OF CONTRACT, TORT OR OTHERWISE, ARISING FROM,
% OUT OF OR IN CONNECTION WITH THE SOFTWARE OR THE USE OR OTHER DEALINGS IN
% THE SOFTWARE.
%-------------------------------------------------------------------------------

%-------------------------------------------------------------------------------
%	DOCUMENT CONFIGURATIONS
%-------------------------------------------------------------------------------

\documentclass[
    version=last,           % use last release of scrlttr2 package
    DIV=13,                 % ?
    BCOR=0mm,               % ?
    paper=a4,               % paper size
    fontsize=12pt,          % fontsize
    firsthead=on,           % display header on first page
    firstfoot=on,           % display footer on first page
    pagenumber=on,i         % position of the page number
    parskip=half,           % Use indent instead of skip, half, false
    enlargefirstpage=,      % more space on first page
    firsthead=on,           % Display the adress
    fromrule=afteraddress,  % separate the address with a line in letter he
    priority=off,           % print a priority onto the document
    backaddress=true,       % Adress to send back
    refline=dateright,      % location of the date
	fromalign=right,	    % Aligns the from address to the right
    fromemail=on,i          % turn on email of sender
    fromurl=on,             % print URL of sender
    frombank=on,
    fromphone=on,           % turn on phone of sender
    frommobilephone=on      % Mobile Phone number
    fromlogo=on,            % turn on logo of sender
    addrfield=on,           % address field for envelope with window, on or true
    subject=untitled,  % placement of subject, beforeopening or titled
    foldmarks=off,          % print foldmarks
    numericaldate=off,      % display date in numbers only
	pagenumber=right,	        % Set page numbers from page 2 onwards
	parskip=half,	        % Separates paragraphs with some whitespace,
    headsep=false,          % Seperatorline in the header
    footsepline=true,       % Seperatorline in the footer
			%use parskip=full for more space or comment out to return to default
    foldmarks=off,		    % Prints small fold marks on the left of the page
	]{scrlttr2}

% Set Font: sans serif Latin Modern
\usepackage{lmodern}
\usepackage[T1]{fontenc} % For extra glyphs (accents, etc)
\usepackage{stix} % Use the Stix font by default
\usepackage{lmodern}
\usepackage[utf8]{inputenc}
%\usepackage[english]{babel}
\usepackage[ngerman,english]{babel}
\usepackage{url}
\usepackage{csquotes}
\usepackage{listings}
\usepackage{verbatim}
\usepackage{graphicx}
\usepackage{tabularx}
\renewcommand*{\raggedsignature}{\raggedright} %Signatur rechtsbündig

\usepackage{array}
\usepackage{makecell}


% Set Page layout:
\usepackage{changepage}
    %\changepage{text height}{text width}{even-side margin}
    %{odd-side margin}{column sep.}
    %{topmargin}{headheight}{headsep}{footskip}
\changepage{}{}{}{}{}{}{}{}{}

%-------------------------------------------------------------------------------
%	YOUR INFORMATION AND LETTER DATE
%-------------------------------------------------------------------------------

\KOMAoptions{foldmarks=true,
foldmarks=false,
fromurl=false,
fromemail=true,
fromphone=false,
fromfax=false,
fromalign=right,
fromrule=off,
footsepline=on,
fromlogo=true,
headsepline = true,
footsepline = true
%fromrule=afteraddress%  % Trennlinie unter dem Briefkopf
}

%\setkomavar{fromlogo}{\parbox[b]{8cm}{\usekomafont{fromaddress}%
%        {\mbox{\LARGE \bfseries Embedded System Design GmbH}}
%        \smallskip}
%}
%

\setkomavar*{urlseparator}{}
\setkomavar{urlseparator}{}
\setkomavar{frombank}{Eine Bank\\BLZ~123\,45\,678\\Kto~123456789}
\setkomavar{fromname}{Marco Israel} % Your name used in the from address
\setkomavar{fromurl}{www.marcois.eu}
\setkomavar{fromaddress}{Am Wickenkamp 38\\ 32351 Stemwede}
\setkomavar{fromemail}{Marco-Israel@online.de}
\setkomavar{signature}{Marco Israel}
\setkomavar{fromphone}{05773 00000000}
\setkomavar{frommobilephone}{0175 00000}
%\setkomavar{place}{Stemwede}
%\includegraphics[⟨options⟩]{⟨file⟩}
\firstfoot{}
\nextfoot{}


%-------------------------------------------------------------------------------
%  HEADER SECTION
%-------------------------------------------------------------------------------
%\firsthead{\centering
%        {\mbox{\LARGE \bfseries Embedded System Design GmbH}}
%}
%

%-------------------------------------------------------------------------------
%  FOOTER SECTION
%-------------------------------------------------------------------------------
%\firstfoot{%
%\centering
%{\renewcommand{\\}{\ {\large\textperiodcentered}\ }
%\small\usekomavar{frombank}
%}%
%}
%




%-------------------------------------------------------------------------------

\begin{document}

%-------------------------------------------------------------------------------
%	ADDRESSEE
%-------------------------------------------------------------------------------

% Addressee name and address
\begin{letter} {Wilhelm Büchner Hochschule \\
Hilpertstr. 31\\
64295 Darmstad}

\setkomavar*{yourref}{AufgabenCode}
\setkomavar{yourref}{CLE01-XX1-N01}		% Ihr Zeichen
\setkomavar*{yourmail}{HeftKürzel} 	% Ihr Schreiben vom
\setkomavar{yourmail}{CLE01XX 1} 	% Ihr Schreiben vom
\setkomavar*{myref}{Auflage}       	% Unser Zeichen
\setkomavar{myref}{0311 N01}       	% Unser Zeichen
\setkomavar*{customer}{Matrikel-Nr}
\setkomavar{customer}{580201} 	% Kundennummer
\setkomavar*{invoice}{StudiengangsNr.}   	% Rechnungsnummer
\setkomavar{invoice}{1640}   	% Rechnungsnummer
\setkomavar{place}{Stemwede}		% Ort
\setkomavar{date}{\today}			% Datum

\setkomavar{subject}{Einsendeaufgaben Typ A }

\opening{Sehr geehrte(r) Herr / Frau}

Guten Tag,

im Anhang die Lösungen für o.g. Einsendeaufgabe Typ A,
\\
\begin{itemize}
    \item Multiple Choice
        \begin{enumerate}
        \item \textbf{Collaboration:} C
        \item \textbf{Gruppen, Teams, Communities und Organisationen:} B
        \end{enumerate}
            \vspace{1cm}
    \item Textaufgaben
        \begin{enumerate}
        \setcounter{enumi}{2}
    \item \textbf{Drei Ebenen der Zusammenarbeit}
        \begin{description}
    \item[Sprint:]  Ansammlung individueller Arbeiten die von einzelnen ohne
        Interaktion bearbeitet werden. Dabei ist keine Koordination zwischen den
        einzelnen Personen und ihren Aufgaben nötig.
    \item[Staffel:] Die Arbeiten sind in Pakete geteilt, die individuell und
        einzeln / alleine bearbeitet werden. Nach vollendung des Arbeitspaketes
        ist eine Übergabe nötig, damit der nächste Mitmensch an der Aufgabe
        weiterarbeiten und um sein Arbeitspaket ergänzen kann. Am Ende steht ein
        Ergebnis, das aus vielen einzelarbeiten zusammen gesetzt wurde. Hierbei
        ist lediglich die Übergabe und die Schnittstellen zwischen den Arbeiten
        abzustimmen und zu koordinieren.
    \item[Mannschaft:] Hierbei ist die stetige Zusammenarbeit einer
        Mahnschaft/eines Teams / Organisation ... nötig. Keiner kann das Ziel
        alleine erarbeiten/erreichen und jedes Team-Mitglied hat den gleichen
        (stellen) Wert. Hier muss das Team stetig koordiniert, synchronisiert
        und zum Ziel ausgerichtet werden.
        werden.
        \end{description}

            \vspace{1cm}
        \item \textbf{Sechs Muster der Zusammenarbeit}
            \begin{description}
        \item[Generierung:] Dieses Muster dient der Generierung und Sammlung von
            Ideen innerhalb einer Zusammenarbeit. Der Ideenpool wird durch
            Kreativität / Kreativitätstechniken und/oder das Sammeln von
            Informationen und den Austausch von Gedanken erweitert werden.
        \item [Reduktion:] Aus einer Vielzahl von Ideen werden diejenigen
            herausgearbeitet, welche in der Gruppe am meisten Aufmerksamkeit
            bekommt. Durch das Reduzieren der Ideen wird zunehmend auch auf
            die wesentlichen Informationen reduziert oder abstrahiert und so der
            Kern verschiedener Ideen zu einer klaren Idee zusammen gefasst. Das
            Reduzieren bedeutet nicht, das sich auf Ideen geeinigt wurde. Es ist
            lediglich ein Reduzieren und Zusammenfassen des Gesamten, auf wenige
            in sich eigenständig stimmige Ideen.
        \item [Verdeutlichung:]
            Es soll ein gemeinsames Verständnis über die verbliebenen
            (reduzierten, zusammengefassten, abstrahierten) Ideen in der Gruppe
            erreicht werden. Hierzu muss ein gemeinsames Verständnis über die
            Ideen und deren Sinn / Bedeutung geschaffen werden. Damit einher
            geht auch (Voraussetzung für das Verdeutlichen ist) ein Verständnis
            des eigentlichen Problems / der Herausforderung und seiner
            Lösungsalternativen.
        \item [Gliederung und Organisierung:]
            Hier soll ein Verständnis über bestehende oder mögliche Beziehungen
            zwischen den Ideen erreicht werden. So werden auch komplexe Ideen
            verständlicher. Die Gliederung und organisation der verbleibenden
            Ideen erleichtert zudem die nachfolgenden Arbeitsschritte.
        \item [Bewertung:]
            Das Bewerten von Ideen setzt zunächst voraus, das ein Verständnis
            der Idee(n) und jener Werte entwickelt wird. Dieses Verständnis
            unterstützt letzten Endes bei der Entscheidungsfindung und hilft der
            Gruppenkommunikation. Hierbei werden zunächst individuelle
            Präferenzen begründet und anschließend zu einer Gruppenpräferenz
            weiterentwickelt. Durch die Bewertung werden auch neue Thesen,
            Meinungsunterschiede und -Übereinstimmungen aufgedeckt.
         \item [Einigung / Konsensbildung:]
             Durch verschiedene Techniken wie dem herausarbeiten der
             gemeinsamen Präferenzen oder der Klärung der Gegensätze soll eine
             Einigung auf Ideen geschaffen werden, welche als Lösung(en) zur
             Aufgabenstellung dienen.
            \end{description}

            \vspace{1cm}
        \item \textbf{Fünf Effekte der Zufriedenheit in der IS-Forschung (aus
            zehn Effekten) }
            In der IS-Forschung wird wischen zwei unterschiedlichen Bedeutungen
            unterschieden:
            \begin{description}
        \item[Zufriedenheit als Urteil:] Hier ist Zufriedenheit demnach erfüllt,
            inwieweit ein Bedüfnis erfüllt wurde. Beispielsweise ein
            Informationsbedürfnis, Hunger, Schlaf oder Leistungen eines
            Arbeitgebers.
        \item [Zufriedeneheit als Gemütserregung:] Dies wird als emotionale
            Reaktion beschrieben. Beispielsweise bei einer wahrgenommenen,
            erfüllten Verantwortung gegenüber etwas.
            \end{description}
            Die Yield Shift Theory of Satisfaction grenzt dabei zehn
            wiederkehrende Muster der Zufriedenheit ab. Beispielsweise:
            \begin{description}
            \item [Nostalgic effects:] Dies ist eine positive oder negative
                Zufriedenheit basierend auf vergangenen Erfolgen oder
                Misserfolgen ohne das sich diese alten Umstände auf
                gegenwärtige Umstände auswirken.
            \item [Differential effects:] Trotz das unterschiedliche Individuen
                ähnlichen Nutzen aus einer Sache ziehen, empfinden sie ein
                unterschiedliches Gefühl der Zufriedenheit / Unzufriedenheit
                (trotz gleicher Erfolge / gleichem Nutzen).
            \item[Hygiene effect:] Hierbei wird davon ausgegangen, dass
                Individuen einen IS-/IT-Gegenstand niemals zustimmen, sondern
                im besten fall neutral oder sonst ablehnend gegenüberstehen,
                unabhängig ob des Objekt fehlerfrei ist.
            \item [Mixed feelings:] Dies beschreibt ein gemischtes Gefühl von
                Individuen gegenüber neuem. Beispielsweise der Neueinführung
                einer Software durch die IT.
            \item [Attenuation effects:] Dies beschreibt, das eine anfängliche
                Zufriedenheit oder Unzufriedenheit über die Zeit abnimmt /
                abschwächt (gedämpft wird).
            \end{description}

            \vspace{1cm}
        \item \textbf{Yield Shift Theory of Satisfaction nach Briggs}:
            Die Yield Shift Theory of Satisfaction beschreibt Erklärungsansätze
            zur Entstehung und Veränderung von (Un-) Zufriedenheit. Die Theorie
            grenzt dabei zehn wiederkehrende Muster der Zufriedenheit ab
            (Beispiele s.o).
            \\
            Der Effekt in der Aufgabenstellung lässt sich darin begründen, das
            Gruppe A nicht trotz der IT-Probleme Zufriedener zu sein schien,
            sonder wegen der kürzlichen (noch präsenten) Probleme zufriedener
            ist. Die Probleme zeigen den Mitarbeitenden, das die (Berufs-)Welt
            nicht perfekt ist, aber sie einen Arbeitgeber oder eine IT-Abteilung
            haben, welche Sie nicht im Stich lässt, sondern die Probleme (für
            sie) löst. Die Mitarbeiter sind dankbar, das Ihnen geholfen wurde
            und sie ihrer Arbeit wieder nachgehen, welche sie aus verschiedenen
            intrinsischen und/oder extrinsischen Gründen tun (möchten). Die
            Mitarbeiter wissen aus ihrem privaten Leben (insgeheim oder
            bewusst), dass auch dort immer wieder Herausforderungen auf sie
            warten, welche Ihnen dort nicht so einfach abgenommen werden. Sie
            werden durch die Probleme an ihre eigenen erinnert und sind dankbar,
            dass dieses im Berufsleben für sie gelöst werden.
            Werden solche und andere Probleme wiederkehrend unkompliziert für
            sie von anderen gelöst, tritt ein \textit{Nostalgic effect} ein
            (s.o.).
            \\
            Bei Gruppe B
            tritt so etwas wie ein \textit{Attenuation effect} auf (s.o), bei
            welchem die laufende IT Infrastuktur als selbstverständlich
            angesehen wird und alle anderen Problems des Alltages weiter in den
            Vordergrund rücken, welche die Gesamt-Zufriedenheit stört.

            \vspace{1cm}
            \item \textbf{ Fünf Ursachen für Konflikte und mögliche Lösungsansätze }
            \vspace{0.5cm}
                \\
            \begin{tabular}{|c|c|c| }
            \hline  \textbf{ Konflikt Art} & \textbf{ Konflikt Beschreibung}
                                           & \textbf{Lösung Beschreibung}\\
                \hline Bedeutungsunterschiede & \thead{Unterschiedliches Verständnis
                \\ einer Sache } & \thead{ Definieren der Sache \\ und ihrer Beziehungen }\\
                \hline \thead{Widersprüchliche \\ Information } & \thead{
                Asymmetrische Informationen \\ führen zu Konflikten } & \thead{
                Informationen Gegenüberstellen, \\ Vergleichen und auf
             \\Glaubwürdigkeit prüfen} \\
             \hline \thead{Unterschiedliche \\ mentale Modelle \\ (Ursache-Wirkung Modelle) }
             & \thead {Mentale Modelle können \\ durch Unstimmigkeit zu \\ Problemen führen }  &
             \thead{ Die Unterschiede untersuchen, \\ vergleichen und
                 hinterfragen \\ oder andere Modelle wählen } \\
                \hline \thead{ Inkompatible individuelle \\ Ziele} & \thead { Die
                Ausschließlichkeit \\ der eigenen und mit anderen \\ inkompatiblen
                Ziele \\ führen zu Konflikten } & \thead{ Wahrscheinlichkeiten und Nutzen \\
                jener Ziele diskutieren \\ und/oder alternativen finden } \\
                \hline Geschmacksunterschiede &  \thead{(persönliche oder
                wissensbasierte ) \\ unterschiedliche Favorisierungen  \\ führen
            zu Konflikten } & \thead { Die Unterschiede und Gründe \\
        austauschen und Kompromisse \\ und alternativen endwicken} \\
            \hline

                \end{tabular}

        \end{enumerate}
\end{itemize}
\vspace{1cm}
Mit freundlichen Grüßen \\
\\
    \includegraphics[scale=0.70]{../unterschrift.png}
    \\
    Marco Israel



%\encl{Myattachments} % Attached documents

%-------------------------------------------------------------------------------
%   Postskriptum
%
%\ps\ PS:\ someText
%-------------------------------------------------------------------------------


%-------------------------------------------------------------------------------
%  Anlagen
%\setkomavar*{enclseparator}{Anlage}
%\encl{%
%  Anlage 1\\
%  Anlage 2%
%}
%-------------------------------------------------------------------------------


%-------------------------------------------------------------------------------
%   Verteiler
%\setkomavar*{ccseparator}{Kopie an}
%\cc{%
%  Verteiler 1\\
%  Verteiler 2%
%}
%-------------------------------------------------------------------------------

\end{letter}

\end{document}

%--- EOF -----------------------------------------------------------------------
%
%
