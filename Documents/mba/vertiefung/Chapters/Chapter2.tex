% Chapter 1

\chapter{Einleitung} % Main chapter title
\label{chp:einleitung} % For referencing the chapter elsewhere, use \ref{Chapter1}



\section{Die Problemstellung}
\label{sec:einleitung_problemstellung}
In der Entwicklung, hier gezielt in der Softwareentwicklung, haben sich über die
Jahre verschiedene Vorgehensmodelle mit teils unterschiedlichen Entwicklungsschritte
entwickelt; sei es \gls{agilSW} oder \gls{klassischeSW} Vorgehen.
Unternehmen nutzen diese Modelle, verzichten jedoch zum Teil (nahezu)
vollständig auf einzelne Schritte und kürzen ein Modell aus unterschiedlichen,
zum Teil gerechtfertigten Gründen, ab. Ken Kent Beck (* 1961, ein
US-amerikanischer Softwareentwickler und Berater, sowie  einer der Begründer von
Extreme Programming (XP) und Autor mehrerer Bücher) weiß, das
Entwicklungsprozesse auf das Projekt und Unternehmen zugeschnitten werden
müssen. Die überwiegende Anzahl Vorgehensmodelle gibt solche Flexibilität als
Modell nicht her.


\section{Ziel dieser Arbeit}
Ziel dieser Arbeit ist es, die Entwicklungsschritte je Vorgehensmodelle
darzustellen und zu Kategorisieren. Entwicklungsschritte der Modelle werden
sich dabei in Aufgabe oder dem Vorgehen überscheiden. Diese Überscheidungen
bilden eine Kategorie oder Gruppe.
\\
Dabei werden die einzelnen Vorgehensmodelle umschrieben und ihre
Schritte im Original dargestellt. Der Leser erhält einen Überblick über
bearbeitete Vorgehensmodelle in der Software Entwicklung. Für weitere Details
einzelner Vorgehensmodelle wird dann auf die verwendete Literatur verwiesen.
Ziel dieser Arbeit ist es nicht, die einzelnen Vorgehensmodelle im Detail neu zu
beschreiben.
\\

Im zweiten Hauptteil entsteht aus jeder Kategorie ein \Gls{ThinkLet}.
Dazu werden die Entwicklungsschritte je Vorgehensmodell und Gruppe in der
\Gls{ThinkLet-Sprache} neu beschrieben. Es entstehen so einheitlich beschriebene
Entwicklungsschritte, die für mehrere Vorgehensmodelle gelten.

Die so entstandene Sammlung von ThinkLet-\textit{Bausteinen} kann verwendet
werden, um Entwicklungsprozesse zu definieren, die auf der einen Seite zwar in
ihren Schritten selbst standardisiert sind. Auf der anderen Seite kann der
Entwicklungsprozess aber dynamisch einem Projekt, einem Unternehmen oder etwa
einer Branche angepasst werden. Dabei gilt, dass nicht nur die Entwicklung eines
Produktes \textit{agil} sein muss, sonder
auch der Entwicklungsprozesse selbst \gls{agil} sein darf, solange wie er
einheitlich definierten Mustern (thinkLets) folgt.
\\

Oberbegriffe wie \textit{Vorgehensmodelle} oder \textit{Entwicklungsschritte}
beziehen sich in dieser Arbeit auf die Software Entwicklung. Das in dieser
Arbeit verwendete Vorgehen ließe sich sicher auf andere Entwicklungsbereiche
übertragen.


